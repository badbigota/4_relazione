% Marco l'Eccellente Dio della Modestia
% !TeX encoding = utf8
% !TeX program = pdflatex
% !TeXpellcheck = it_IT

\documentclass[a4paper,11pt,oneside]{article} 

\usepackage{relazioni}
\usepackage{imakeidx}
\usepackage{colortbl}
\usepackage{booktabs}
\usepackage{blindtext}
\usepackage{titletoc}
\usepackage{hyperref}
\usepackage{graphicx}
\usepackage{subcaption}
\usepackage{subfig}
\usepackage{wrapfig}
\usepackage{geometry}
\usepackage{array}
\usepackage[export]{adjustbox}
\usepackage{multirow}
\usepackage{multicol}

\usepackage{rccol}
\usepackage[export]{adjustbox}
\hypersetup{
%    colorlinks=false,
} 

\graphicspath{{Figure/}} 
%https://www.overleaf.com/learn/latex/Indices
%\makeindex[columns=3, title=Alphabetical Index, intoc]

\setlength{\parindent}{0em}


\begin{document}
\input{Front-matter/Frontespizio}

\clearpage

\tableofcontents
\addtocontents{toc}{~\hfill{Pagina}\par}
\contentsmargin{6em}
\dottedcontents{section}[1em]{\bigskip}{2em}{1pc}
\dottedcontents{subsection}[3em]{\smallskip}{3em}{1pc}
\dottedcontents{subsubsection}[5em]{\smallskip}{4em}{1pc}

\newpage

\section{Obiettivo}
L'obiettivo dell'esperienza è la stima della viscosità di un liquido saponoso di cui si conosce la densità.

\section{Apparato Sperimentale}


\begin{wrapfigure}[16]{l}{4cm}
    \caption{Apparato \\Sperimentale}
    \label{fig:apparato_sperimentale}
    \includegraphics[width=4cm]{ApparatoSperimentale.jpg}
\end{wrapfigure}

L'apparato sperimentale risulta così composto:
\begin{enumerate}[label={\alph*.}]
    \item Dispositivo magnetico per il rilascio delle sferette metalliche posto sopra l'imboccatura superiore
    \item Cilindro in plastica trasparente (dal diametro di $\SI{4.5}{cm}$ e altezza di $\approx\SI{70}{cm}$)
    \item Liquido saponoso di densità ${\rho}_L=(1.032\pm0.001)\si{\gram\per\cubic\centi\metre}$, contenuto nel cilindro fino ad un livello di $\approx\SI{10}{cm}$ superiore alla tacca di referenza più alta
    \item 11 tacche incise sul cilindro tra loro distanziate di $\SI{5}{cm}$
    \item 10 sferette in acciaio di densità $\rho_S=(7.870\pm0.005)\si{\gram\per\cubic\centi\metre}$ per ogni misura di diametro disponibile come riportato in Tabella \ref{tab:diametri_sfere}
    \item Videocamera a 29.97 fps per registrare tutte le cadute delle sferette nel fluido 
\end{enumerate}
In questa tabella vengono presentate le caratteristiche delle sfere utilizzate durante tutta l'esperienza.



%AGGIUNGERE ERRORE DIAMETRO
%\begin{table}[h!]
%\centering
%\begin{tabular}{|c|c|} 
%\hline
%\textbf{N pallina } & \textbf{Diametro } \\ 
%\hline
%\rowcolor[rgb]{0.85,0.85,0.85} \textbf{1 } & 1.5 mm\\ 
%\hline
%\textbf{2 } & 2”/32 \\ 
%\hline
%\rowcolor[rgb]{0.85,0.85,0.85} \textbf{3 } & 2.0 mm \\
%\hline
%\textbf{4 } & 3”/32 \\ 
%\hline
%\rowcolor[rgb]{0.85,0.85,0.85} \textbf{5 } & 4”/32 \\ 
%\hline
%\textbf{6 } & 5”/32 \\ 
%\hline
%\rowcolor[rgb]{0.85,0.85,0.85} \textbf{7 } & 6”/32 \\ 
%\hline
%\textbf{8 } & 7”/32 \\ 
%\hline
%\rowcolor[rgb]{0.85,0.85,0.85} \textbf{9 } & 8”/32 \\ 
%\hline
%\textbf{10 } & 9”/32 \\
%\hline
%\end{tabular}
%\captionsetup{labelformat=empty}
%\caption{Sferette Ebbasta}
%\label{tab:diametri_sfere}
%\end{table}
\bigskip
\bigskip
\begin{table}[h!]
    \centering
    \makebox[\textwidth]{
    \begin{tabular}{|c|c|c|c|c|c|c|c|c|c|c|}
        \hline
        \textbf{N pallina} & $1$ & $2$ & $3$ & $4$ & $5$ & $6$ & $7$ & $8$ & $9$ & $10$ \\ \hline
        \textbf{Diametro} $\pm 0.01$ mm & $1.5$ mm & $2"/32$ & $2$ mm & $3"/32$ & $4"/32$ & $5"/32$ & $6"/32$ & $7"/32$ & $8"/32$ & $9"/32$ \\ \hline
    \end{tabular}}
    \caption{Diametri sfere}
    \label{tab:diametri_sfere}
\end{table}

Si specifica che la sfera 9 è stata usata per il nono e decimo lancio, mentre la sfera 10 è stata impiegata solo nell'undicesimo lancio.

\section{Presa Dati}
I dati sono stati raccolti da 3 operatori analizzando i video forniti.
Sono state utilizzate due metodologie differenti:

\paragraph{Presa dati tramite software VLC}
L'operatore A e l'operatore B hanno compiuto la presa dati tramite un'estensione del software VLC che visualizza il tempo corrispondente al singolo frame visualizzato con sensibilità di $\SI{1e4}{\second^{-1}}$. In ogni video si è scelto di annotare la misurazione temporale nel momento in cui il centro della sfera era in corrispondenza del livello segnato dalla tacca incisa sul viscosimetro. Si specifica che il video relativo alla nona sfera è stato analizzato soltanto tramite questa modalità in quanto non si disponeva di sufficienti misure.

\paragraph{Presa dati tramite cronometro manuale}
Il terzo operatore (C) ha invece eseguito la presa dati tramite un cronometro manuale. Quest'ultimo è stato avviato in corrispondenza del passaggio della sferetta con la prima tacca di lettura del cilindro e sono stati trascritti i tempi parziali al passaggio di ciascuna tacca successiva. Si è posta particolare attenzione nel leggere i tempi dal cronometro nell'istante in cui il centro della sferetta passasse in corrispondenza della tacca incisa. Per ogni viscosimetro analizzato in questo modo sono state prese le misure 5 volte.\\ \newline
Ogni operatore ha analizzato video differenti ad esclusione del video che riguardava la sesta pallina, che sono stati analizzati da tutti e tre gli operatori e del nono lancio, analizzato dagli operatori A e B.

\section{Analisi}

\subsection{Tempi di passaggio}
%TABELLONE GRANDE DATI GREZZI, attenzione, bisogna unire le colonne valore-errore
\begin{table}[h!]
\footnotesize
\makebox[\textwidth]{
\begin{tabular}{|c|c|c|c|c|c|c|c|}
\hline
& 1 & 2 & 3 & 4 & 5 & 7 & 8 \\
& \si{[ms]} & [ms] & [ms] & [ms] & [ms] & [ms] & [ms] \\ \hline
{\rowcolor[rgb]{0.85,0.85,0.85}}$1^o$ & $63415 \pm 61$ & $57453 \pm 54$ & $0 \pm 0$ & $24020 \pm 27$ & $26174 \pm 20$ & $0 \pm 0$ & $0 \pm 0$ \\ \hline
$2^o$ & $102383 \pm 61$ & $92344 \pm 54$ & $22064 \pm 85$ & $38755 \pm 27$ & $34120 \pm 20$ & $3102 \pm 13$ & $1672 \pm 34$ \\ \hline
{\rowcolor[rgb]{0.85,0.85,0.85}}$3^o$ & $141856 \pm 61$ & $128619 \pm 54$ & $44056 \pm 101$ & $53411 \pm 27$ & $42461 \pm 20$ & $6284 \pm 22$ & $3578 \pm 53$ \\ \hline
$4^o$ & $181104 \pm 61$ & $163108 \pm 54$ & $66034 \pm 59$ & $68254 \pm 27$ & $50648 \pm 20$ & $9486 \pm 14$ & $5382 \pm 44$ \\ \hline
{\rowcolor[rgb]{0.85,0.85,0.85}}$5^o$ & $221178 \pm 61$ & $198367 \pm 54$ & $88114 \pm 55$ & $83044 \pm 27$ & $58820 \pm 20$ & $12628 \pm 31$ & $7320 \pm 54$ \\ \hline
$6^o$ & $261074 \pm 61$ & $234132 \pm 54$ & $110248 \pm 60$ & $97691 \pm 27$ & $66990 \pm 20$ & $15728 \pm 53$ & $9206 \pm 48$ \\ \hline
{\rowcolor[rgb]{0.85,0.85,0.85}}$7^o$ & $301213 \pm 61$ & $269735 \pm 54$ & $132260 \pm 112$ & $112475 \pm 27$ & $75233 \pm 20$ & $18784 \pm 22$ & $11102 \pm 35$ \\ \hline
$8^o$ & $341171 \pm 61$ & $305387 \pm 54$ & $154530 \pm 118$ & $127363 \pm 27$ & $83368 \pm 20$ & $22146 \pm 44$ & $13014 \pm 25$ \\ \hline
{\rowcolor[rgb]{0.85,0.85,0.85}}$9^o$ & $381196 \pm 61$ & $340954 \pm 54$ & $176914 \pm 51$ & $142164 \pm 27$ & $91720 \pm 20$ & $25234 \pm 17$ & $14870 \pm 27$ \\ \hline
$10^o$ & $421205 \pm 61$ & $376724 \pm 54$ & $199198 \pm 66$ & $157075 \pm 27$ & $99974 \pm 20$ & $28440 \pm 41$ & $16890 \pm 53$ \\ \hline
{\rowcolor[rgb]{0.85,0.85,0.85}}$11^o$ & $461549 \pm 61$ & $412758 \pm 54$ & $221638 \pm 29$ & $172167 \pm 27$ & $108361 \pm 20$ & $31612 \pm 18$ & $18728 \pm 33$ \\ \hline\hline

& 6_A & 6_B & 6_C & 9_A & 9_B & 10 & 11\\
& \si{[ms]} & [ms] & [ms] & [ms] & [ms] & [ms] & [ms]\\ \hline
{\rowcolor[rgb]{0.85,0.85,0.85}}$1^o$ & $7005 \pm 20$ & $6987 \pm 20$ & $0 \pm 0$ & $2397 \pm 14$ & $2353 \pm 14$ & ND & ND \\ \hline
$2^o$ & $12012 \pm 20$ & $12021 \pm 20$ & $4932 \pm 34$ & $3138 \pm 19$ & $3066 \pm 19$ & $10521 \pm 14$ & $9908 \pm 14$ \\ \hline
{\rowcolor[rgb]{0.85,0.85,0.85}}$3^o$ & $17111 \pm 20$ & $17103 \pm 20$ & $10044 \pm 29$ & $3943 \pm 14$ & $3989 \pm 14$ & ND & ND \\ \hline
$4^o$ & $22048 \pm 20$ & $22064 \pm 20$ & $15004 \pm 53$ & $4906 \pm 14$ & $4875 \pm 14$ & ND & ND \\ \hline
{\rowcolor[rgb]{0.85,0.85,0.85}}$5^o$ & $27210 \pm 20$ & $27205 \pm 20$ & $20072 \pm 58$ & $5858 \pm 14$ & $5805 \pm 14$ & ND & ND \\ \hline
$6^o$ & $32288 \pm 20$ & $32263 \pm 20$ & $25156 \pm 51$ & $6805 \pm 14$ & $6762 \pm 14$ & ND & ND \\ \hline
{\rowcolor[rgb]{0.85,0.85,0.85}}$7^o$ & $37352 \pm 20$ & $37378 \pm 20$ & $30328 \pm 54$ & $7730 \pm 14$ & $7692 \pm 14$ & ND & ND \\ \hline
$8^o$ & $42461 \pm 20$ & $42460 \pm 20$ & $35376 \pm 22$ & $8715 \pm 19$ & $8656 \pm 19$ & ND & ND \\ \hline
{\rowcolor[rgb]{0.85,0.85,0.85}}$9^o$ & $47635 \pm 20$ & $47603 \pm 20$ & $40560 \pm 31$ & $9711 \pm 19$ & $9657 \pm 19$ & ND & ND \\ \hline
$10^o$ & $52711 \pm 20$ & $52702 \pm 20$ & $45700 \pm 38$ & $10640 \pm 19$ & $10584.5 \pm 19$ & ND & ND \\ \hline
{\rowcolor[rgb]{0.85,0.85,0.85}}$11^o$ & $57902 \pm 20$ & $57944 \pm 20$ & $50810 \pm 62$ & $11625 \pm 19$ & $11578 \pm 19$ & $18871 \pm 14$ & $14058 \pm 14$ \\ \hline
\end{tabular}}
\caption{Tempi estratti dai video}
\label{tab:dati_semi_grezzi}
\end{table}



Vengono riportati in Tabella \ref{tab:dati_semi_grezzi} le misurazioni dei tempi effettuate dai diversi operatori per le diverse sfere in esame riassunti per colonne, in base alla numerazione dei video/lanci.
Per la sesta e per la nona sfera vengono riportate misurazioni differenti ciascuna derivante da un operatore diverso.
La decima ed undicesima sfera riportano soltanto le misure dei tempi di passaggio in corrispondenza della seconda ed ultima tacca in quanto non è stato possibile effettuare altre misurazioni temporali a causa dell'elevata velocità di discesa della sfera nel fluido.
Si precisa che le misure presenti nella Tabella \ref{tab:dati_semi_grezzi} relative all'operatore C rappresentano la media delle misure relative a i 5 tempi di passaggio della sfera per lo stesso traguardo.
Si noti inoltre che le misure effettuate dagli operatori A e B indicano l'istante temporale di passaggio della sfera in corrispondenza della tacca avendo come origine temporale l'inizio del video analizzato. Le misure dell'operatore C invece hanno come origine temporale  del conteggio l'istante di passaggio della sfera per la prima tacca.
%Per una trattazione approfondita degli errori associati alle misure si rimanda alla sezione Discussione.\\


\subsection{Calcolo $\Delta t$}
%DELTA T CON FBF
Per il calcolo di $\Delta t_{i}$ per le sfere analizzate soltanto da un operatore tramite il metodo del frame-by-frame è stato sufficiente eseguire la differenza $t_{i+1}- t_{i}$ associandovi l'errore derivante dal teorema delle varianze.\\
\newline
%DELTA T CON CRONOMETRO
Per il computo dei $\Delta t_{i}$ relativi alle misure eseguite dal terzo operatore si sono calcolati cinque campioni $j$ di $\{\Delta t_{i=1\dots10}\}_{j=1\dots5}$. Ciascuno dei campioni è riferito ad un'unica presa dati relativa ad uno stesso video.
Nel campione riassuntivo di ${\Delta t_{i}}$, $\Delta t_{i=k}$ riassuntivo della singolo intervallo tra tacche è stato ricavato dalla media di $\{\Delta t_{i=k}\}_{j=1\dots5}$ associando a ciascuno di essi l'errore derivante dalla deviazione standard calcolata sulla media.


%SPIEGAZIONE DEL PERCHÈ QUESTO METODO NON NE ACCENNIAMO QUI? CITARE IN DISCUSSIONE - si ne scriviamo nella discussione (Marco)

\subsubsection*{Confronto metodologie impiegate dai 3 operatori - $6^{\degree}$ viscosimetro}%Analisi del viscosimetro 6}
Al fine di valutare la compatibilità tra i diversi metodi di misura e operatori, il viscosimetro 6 è stato analizzato da tutti i membri del gruppo.\\
Si è calcolata la compatibilità $\lambda$ tra ciascun $\Delta t_i$ con i corrispondenti $\Delta t_i$ ottenuti dagli  altri operatori, come viene esposto riportato nella Tabella \ref{tab:Confronto_tutte_le_compatibilità_per_ogni_misura}.
In seguito sono state ricavate delle $\lambda$ relative agli interi campioni, calcolate a partire dalla media e dalla deviazione standard della media di ciascun campione di misure, come mostrato in Tabella \ref{tab:Confronto_tutte_le_compatibilità}. Si è fatto riferimento alle seguenti per valutare $\lambda$ e la sua bontà:
\begin{equation*}%Comp
    \label{eq:cases}
    \begin{cases}
    0<\lambda\leq 1, & \text{Ottima}\\
    1<\lambda\leq2, & \text{Discreta}\\
    2<\lambda\leq3, & \text{Pessima}\\
    3<\lambda, & \text{Non compatibile}\\
    \end{cases}
\end{equation*}


\begin{figure}[h!]
\small
    \centering
    \label{fig:comp}
    \subfloat[Confronto $\lambda$ per i singoli $\Delta t$]{
    \label{tab:Confronto_tutte_le_compatibilità_per_ogni_misura}
    \begin{tabular}{|c|c|c|c|c|c|c|c|c|c|c|}
        \hline
        & 1^a & 2^a & 3^a & 4^a & 5^a & 6^a & 7^a & 8^a & 9^a & 10^a\\ \hline
        \rowcolor[rgb]{0.85,0.85,0.85}A - B & $0.66$ & $0.42$ & $0.59$ & $0.51$ & $0.49$ & $1.2$ & $0.66$ & $0.76$ & $0.56$ & $1.2$\\ \hline
        A - C & $1.7$ & $0.33$ & $0.36$ & $1.0$ & $0.072$ & $1.9$ & $1.2$ & $0.27$ & $1.3$ & $1.2$\\ \hline
        \rowcolor[rgb]{0.85,0.85,0.85}B - C & $2.3$ & $0.75$ & $0.016$ & $0.77$ & $0.31$ & $0.98$ & $0.64$ & $1.1$ & $0.85$ & $2.0$\\  \hline
    \end{tabular}
    }
    \subfloat[Confronto $\lambda$]{
        \label{tab:Confronto_tutte_le_compatibilità}
        \begin{tabular}{|c|c|} 
        \hline
         & $\lambda$   \\ \hline
        \rowcolor[rgb]{0.85,0.85,0.85} A - B & $0.18$  \\ \hline
        A - C & $0.24$   \\ \hline
        \rowcolor[rgb]{0.85,0.85,0.85} B - C & $0.42$  \\ \hline
        \end{tabular}
    }
\end{figure}


Per un confronto visivo è stato realizzato il Grafico \ref{fig:andamento_delta_t} relativo all'andamento dei $\Delta t_{i}$ per ciascun operatore. Si specifica che si è scelto di traslare i vari $\Delta t$ degli operatori al fine di agevolare la lettura del grafico.

\begin{figure}[h!]
    \centering
    \includegraphics[width=1\textwidth]{delta_t_tutti_pdf.pdf}
    \caption{Andamento $\Delta t_{i}$ per il viscosimetro 6}
    \label{fig:andamento_delta_t}
\end{figure}

\subsubsection*{Computo unico campione di  $\Delta t$ per i viscosimetri 6 e 9}
I 3 set di dati di $\{\Delta t_{i=1\dots10}\}_{j=A \dots C}$ relativi a ciascun operatore, riferiti alla pallina avente diametro 5"/32, sono stati riassunti in un unico campione di $\{\Delta t_{i}\}$ dove ogni $\Delta t_{i}$ risultava dalla media ponderata di $\{\Delta t_{i}\}_{j=A \dots C}$. L'errore associato ai singoli ${\Delta t_{i}}$ è stato calcolato tramite l'errore relativo alla media ponderata.\newline
Analogo procedimento è stato compiuto per le misure relative al viscosimetro 9.
Il campione $\{\Delta t\}$ riassuntivo delle varie misure è stato poi impiegato per la stima delle velocità.


\subsection{Verifica del raggiungimento della $v_{lim}$}

Per ciascun viscosimetro si è proceduto al calcolo di un campione di velocità $\{v_{i=1\dots 10}\}$ in cui $v_{i} = \frac{\Delta x}{\Delta t_i}$ con $\Delta t_i$ la differenza tra i tempi appena calcolata e $\Delta x$ la distanza fra due tacche. Si riportano i grafici delle velocità di tutti i viscosimetri, eccetto per i lanci 10 e 11 di cui si riportano le velocità nelle Tabelle \ref{tab:vel10_11}.

%TUTTI I GRAFICHETTI DELLE VELOCITa'
\begin{figure}[h!]
    \centering
    \caption{Viscosimetro 1}
    \makebox[\textwidth]{
    \subfloat{
        \includegraphics[width=9.5cm]{uno.pdf}
        \label{fig:vel1}
    }
    \subfloat{
    \begin{small}
        \begin{tabular}{|c|c|}
        \hline
        t_{int} [$ms$]& v $10^{-5}$ $[\si{\metre / \second}]$\\
        \hline
        \rowcolor[rgb]{0.9,0.9,0.9}$19484$ & $128 \pm 1$\\
        $58705$ & $127 \pm 1$\\
        \rowcolor[rgb]{0.9,0.9,0.9}$98065$ & $127 \pm 1$\\
        $137726$ & $125 \pm 1$\\
        \rowcolor[rgb]{0.9,0.9,0.9}$177711$ & $125 \pm 1$\\
        $217728$ & $125 \pm 1$\\
        \rowcolor[rgb]{0.9,0.9,0.9}$257777$ & $125 \pm 1$\\
        $297768$ & $125 \pm 1$\\
        \rowcolor[rgb]{0.9,0.9,0.9}$337786$ & $125 \pm 1$\\
        $377962$ & $124 \pm 1$\\
        \hline
    \end{tabular}
    \end{small}
    }}
\end{figure}

\begin{figure}[h!]
    \centering
    \caption{Viscosimetro 2}
    \makebox[\textwidth]{
    \subfloat{
        \includegraphics[width=9.5cm]{due.pdf}
        \label{fig:vel2}
    }
    \subfloat{
    \begin{small}
        \begin{tabular}{|c|c|}
        \hline
        t_{int} [$ms$]& v $10^{-5}$ $[\si{\metre / \second}]$\\
        \hline
        \rowcolor[rgb]{0.85,0.85,0.85}$17446$ & $143 \pm 1$\\
        $53029$ & $138 \pm 1$\\
        \rowcolor[rgb]{0.85,0.85,0.85}$88411$ & $145 \pm 1$\\
        $123284$ & $142 \pm 1$\\
        \rowcolor[rgb]{0.85,0.85,0.85}$158796$ & $140 \pm 1$\\
        $194480$ & $140 \pm 1$\\
        \rowcolor[rgb]{0.85,0.85,0.85}$230108$ & $140 \pm 1$\\
        $265718$ & $141 \pm 1$\\
        \rowcolor[rgb]{0.85,0.85,0.85}$301386$ & $140 \pm 1$\\
        $337288$ & $139 \pm 1$\\
        \hline
    \end{tabular}
    \end{small}
    }}
\end{figure}

\begin{figure}[h!]
    \centering
    \caption{Viscosimetro 3}
    \makebox[\textwidth]{
    \subfloat{
        \includegraphics[width=9.5cm]{tre.pdf}
        \label{fig:vel3}
    }
    \subfloat{
    \begin{small}
        \begin{tabular}{|c|c|}
        \hline
        t_{int} [$ms$]& v $10^{-5}$ $[\si{\metre / \second}]$\\
        \hline
        \rowcolor[rgb]{0.85,0.85,0.85}$11032$ & $227 \pm 2$\\
        $33060$ & $227 \pm 2$\\
        \rowcolor[rgb]{0.85,0.85,0.85}$55045$ & $228 \pm 2$\\
        $77074$ & $226 \pm 2$\\
        \rowcolor[rgb]{0.85,0.85,0.85}$99181$ & $226 \pm 2$\\
        $121254$ & $227 \pm 2$\\
        \rowcolor[rgb]{0.85,0.85,0.85}$143395$ & $225 \pm 2$\\
        $165722$ & $223 \pm 2$\\
        \rowcolor[rgb]{0.85,0.85,0.85}$188056$ & $224 \pm 2$\\
        $210418$ & $223 \pm 2$\\
        \hline
    \end{tabular}
    \end{small}
    }}
\end{figure}

\begin{figure}[h!]
    \centering
    \caption{Viscosimetro 4}
    \makebox[\textwidth]{
    \subfloat{
        \includegraphics[width=9.5cm]{quattro.pdf}
        \label{fig:vel4}
    }
    \subfloat{
    \begin{small}
        \begin{tabular}{|c|c|}
        \hline
        t_{int} [$ms$]& v $10^{-5}$ $[\si{\metre / \second}]$\\
        \hline
        \rowcolor[rgb]{0.85,0.85,0.85}$7368$ & $339 \pm 3$\\
        $22063$ & $341 \pm 3$\\
        \rowcolor[rgb]{0.85,0.85,0.85}$36813$ & $337 \pm 3$\\
        $51629$ & $338 \pm 3$\\
        \rowcolor[rgb]{0.85,0.85,0.85}$66348$ & $341 \pm 3$\\
        $81063$ & $338 \pm 3$\\
        \rowcolor[rgb]{0.85,0.85,0.85}$95899$ & $336 \pm 3$\\
        $110744$ & $338 \pm 3$\\
        \rowcolor[rgb]{0.85,0.85,0.85}$125600$ & $335 \pm 3$\\
        $140601$ & $331 \pm 3$\\
        \hline
    \end{tabular}
    \end{small}
    }}
\end{figure}

\begin{figure}[h!]
    \centering
    \caption{Viscosimetro 5}
    \makebox[\textwidth]{
    \subfloat{
        \includegraphics[width=9.5cm]{cinque.pdf}
        \label{fig:vel5}
    }
    \subfloat{
    \begin{small}
        \begin{tabular}{|c|c|}
        \hline
        t_{int} [$ms$]& v $10^{-5}$ $[\si{\metre / \second}]$\\
        \hline
        \rowcolor[rgb]{0.85,0.85,0.85}$3973$ & $629 \pm 6$\\
        $12117$ & $599 \pm 6$\\
        \rowcolor[rgb]{0.85,0.85,0.85}$20381$ & $611 \pm 6$\\
        $28560$ & $612 \pm 6$\\
        \rowcolor[rgb]{0.85,0.85,0.85}$36731$ & $612 \pm 6$\\
        $44938$ & $607 \pm 6$\\
        \rowcolor[rgb]{0.85,0.85,0.85}$53127$ & $615 \pm 6$\\
        $61370$ & $599 \pm 6$\\
        \rowcolor[rgb]{0.85,0.85,0.85}$69673$ & $606 \pm 6$\\
        $77994$ & $596 \pm 6$\\
        \hline
    \end{tabular}
    \end{small}
    }}
\end{figure}

\begin{figure}[h!]
    \centering
    \caption{Viscosimetro 6}
    \makebox[\textwidth]{
    \subfloat{
        \includegraphics[width=9.5cm]{sei.pdf}
        \label{fig:vel6}
    }
    \subfloat{
    \begin{small}
        \begin{tabular}{|c|c|}
        \hline
        t_{int} [$ms$]& v $10^{-5}$ $[\si{\metre / \second}]$\\
        \hline
        \rowcolor[rgb]{0.85,0.85,0.85}$2498$ & $1001 \pm 10$\\
        $7546$ & $981 \pm 10$\\
        \rowcolor[rgb]{0.85,0.85,0.85}$12570$ & $1010 \pm 10$\\
        $17619$ & $971 \pm 10$\\
        \rowcolor[rgb]{0.85,0.85,0.85}$22727$ & $986 \pm 10$\\
        $27812$ & $980 \pm 10$\\
        \rowcolor[rgb]{0.85,0.85,0.85}$32907$ & $983 \pm 10$\\
        $38035$ & $967 \pm 9$\\
        \rowcolor[rgb]{0.85,0.85,0.85}$43170$ & $981 \pm 10$\\
        $48322$ & $961 \pm 9$\\
        \hline
    \end{tabular}
    \end{small}
    }}
\end{figure}

\begin{figure}[h!]
    \centering
    \caption{Viscosimetro 7}
    \makebox[\textwidth]{
    \subfloat{
        \includegraphics[width=9.5cm]{sette.pdf}
        \label{fig:vel7}
    }
    \subfloat{
    \begin{small}
        \begin{tabular}{|c|c|}
        \hline
        t_{int} [$ms$]& v $10^{-4}$ $[\si{\metre / \second}]$\\
        \hline
        \rowcolor[rgb]{0.85,0.85,0.85}$1551$ & $161 \pm 2$\\
        $4693$ & $157 \pm 2$\\
        \rowcolor[rgb]{0.85,0.85,0.85}$7885$ & $156 \pm 2$\\
        $11057$ & $159 \pm 2$\\
        \rowcolor[rgb]{0.85,0.85,0.85}$14178$ & $161 \pm 3$\\
        $17256$ & $164 \pm 3$\\
        \rowcolor[rgb]{0.85,0.85,0.85}$20465$ & $149 \pm 3$\\
        $23690$ & $162 \pm 3$\\
        \rowcolor[rgb]{0.85,0.85,0.85}$26837$ & $156 \pm 3$\\
        $30026$ & $158 \pm 2$\\
        \hline
        \end{tabular}
    \end{small}
    }}
\end{figure}

\begin{figure}[h!]
    \centering
    \caption{Viscosimetro 8}
    \makebox[\textwidth]{
    \subfloat{
        \includegraphics[width=9.5cm]{otto.pdf}
        \label{fig:vel8}
    }
    \subfloat{
    \begin{small}
        \begin{tabular}{|c|c|}
        \hline
        t_{int} [$ms$]& v $10^{-4}$ $[\si{\metre / \second}]$\\
        \hline
        \rowcolor[rgb]{0.85,0.85,0.85}$836$ & $299 \pm 7$\\
        $2625$ & $262 \pm 8$\\
        \rowcolor[rgb]{0.85,0.85,0.85}$4480$ & $277 \pm 11$\\
        $6351$ & $258 \pm 8$\\
        \rowcolor[rgb]{0.85,0.85,0.85}$8263$ & $265 \pm 8$\\
        $10154$ & $264 \pm 6$\\
        \rowcolor[rgb]{0.85,0.85,0.85}$12058$ & $262 \pm 4$\\
        $13942$ & $269 \pm 4$\\
        \rowcolor[rgb]{0.85,0.85,0.85}$15880$ & $248 \pm 5$\\
        $17809$ & $272 \pm 7$\\
        \hline
        \end{tabular}
    \end{small}
    }}
\end{figure}

\begin{figure}[h!]
    \centering
    \caption{Viscosimetro 9}
    \makebox[\textwidth]{
    \subfloat{
        \includegraphics[width=9.5cm]{nove.pdf}
        \label{fig:vel9}
    }
    \subfloat{
    \begin{small}
        \begin{tabular}{|c|c|}
        \hline
        t_{int} [$ms$]& v $10^{-4}$ $[\si{\metre / \second}]$\\
        \hline
        \rowcolor[rgb]{0.85,0.85,0.85}$363$ & $688 \pm 13$\\
        $1159$ & $579 \pm 10$\\
        \rowcolor[rgb]{0.85,0.85,0.85}$2053$ & $541 \pm 9$\\
        $2986$ & $531 \pm 9$\\
        \rowcolor[rgb]{0.85,0.85,0.85}$3933$ & $525 \pm 9$\\
        $4872$ & $539 \pm 9$\\
        \rowcolor[rgb]{0.85,0.85,0.85}$5823$ & $513 \pm 8$\\
        $6810$ & $501 \pm 7$\\
        \rowcolor[rgb]{0.85,0.85,0.85}$7773$ & $539 \pm 7$\\
        $8732$ & $505 \pm 7$\\
        \hline
    \end{tabular}
    \end{small}
    }}
\end{figure}

\newpage

\begin{figure}
    \centering
    \captionsetup[subfloat]{labelformat=empty}
    \caption*{}
    \label{tab:vel10_11}
    \subfloat[\small Viscosimetro 10]{
    \begin{tabular}{|c|c|}
        \hline
        t_{int} [$ms$]& v $10^{-4}[\si{\metre / \second}]$\\
        \hline
        \rowcolor[rgb]{0.85,0.85,0.85} $4175$ & $539 \pm 1$\\
        \hline
    \end{tabular}
    }
    \hspace{1cm}
    \subfloat[\small Viscosimetro 11]{
    \begin{tabular}{|c|c|}
        \hline
        t_{int} [$ms$]& v $10^{-4}[\si{\metre / \second}]$\\
        \hline
        \rowcolor[rgb]{0.85,0.85,0.85} $2075$ & $1084 \pm 5$\\
        \hline
    \end{tabular}
    }
\end{figure}


%Analizzando i grafici si è osservato l'andamento delle misurazioni. Al fine di poter stimare una velocità limite da cui poi derivare la viscosità $\eta$ si è dapprima verificato se le misure seguissero un andamento di tipo lineare, d'ora in poi assunta come ipotesi nulla.
%Assumendo una densità del fluido costante, l'equazione del moto prevede un andamento della velocità di tipo esponenziale, che si assesta dopo un tempo caratteristico $\approx 3\tau$ ad un andamento di tipo lineare. Tale affermazione è giustificata tenendo in considerazione la variazione delle misurazioni dovuta alla componente di errore casuale, le caratteristiche fisiche delle sfere impiegate e le caratteristiche del fluido stesso.\newline
Si sono impiegati vari metodi per determinare il tipo di andamento delle velocità al variare del tempo e se la velocità limite prevista dalla teoria fosse stata raggiunta per tutte le sfere considerate.

\subsubsection*{Calcolo di t-Student per la stima del coefficiente di correlazione $r$}
%1a con tstudent PDF
Si è calcolato il coefficiente di correlazione $r$ tra tempi intermedi e velocità statisticamente indipendenti $\{ v_{i=2n+1}\}$, al fine di valutare una correlazione statistica di queste ultime con il tempo.
Ponendo come ipotesi nulla $H_{0}$ che $\rho=0$, si è eseguito il test di t-Student a due code determinando per ogni sfera il livello di confidenza entro il quale veniva rigettata l'ipotesi di non-correlazione. \newline
Come si riscontra in Tabella \ref{tab:t_student} si notano percentuali di rigetto dell'ipotesi nulla  mai inferiori al 60\%.
Il dato appena riportato suggerisce che le possibili cause della variazione delle velocità sono o il mancato raggiungimento della velocità limite o la non omogeneità del fluido. Ipotizzando una viscosità del fluido omogenea, si è proceduto alla verifica del raggiungimento della velocità limite tramite i due metodi descritti in seguito.
%Se $\rho$ fosse risultato pari a $0$, ciò avrebbe implicato una indipendenza statistica delle velocità dal tempo, e dunque una costanza delle velocità.
%Infatti il variare del tempo non avrebbe su di esse influito in quanto in linea teorica ci si aspetta che la velocità limite sia stata raggiunta.
%L'ipotesi nulla da verificare è la non correlazione delle velocità rispetto al tempo, ovvero $\rho=0$.



\begin{table}[h!]
\caption{t-Student su $r$}
\label{tab:t_student}
\vspace{0.2cm}
\centering
\begin{tabular}{|c|c|c|c|}
\hline
 \textbf{N pallina} & \textbf{$r$} & \textbf{ variabile t} & \textbf{{$\%$ Rigetto ipotesi}} \\ \hline
\rowcolor[rgb]{0.85,0.85,0.85}$1$ & $-0,9266263744$ & $-4,268708947$ & $96$ \\ \hline
 $2$ & $-0,7808960208$ & $-2,165256859$ & $80$ \\ \hline
\rowcolor[rgb]{0.85,0.85,0.85}$3$ & $-0,8770279297$ & $-3,161807619$ & $90$ \\ \hline
 $4$ & $-0,5612729826$ & $-1,17462206$ & $60$ \\ \hline
\rowcolor[rgb]{0.85,0.85,0.85}$5$ & $-0,7682053355$ & $-2,078355718$ & $80$ \\ \hline
 $6$ & $-0,8377594445$ & $-2,657393791$ & $90$ \\ \hline
\rowcolor[rgb]{0.85,0.85,0.85}$7$ & $-0,5491597001$ & $-1,138150848$ & $60$ \\ \hline
 $8$ & $-0,9685935276$ & $-6,747060187$ & $99$ \\ \hline
\rowcolor[rgb]{0.85,0.85,0.85}$9$ & $-0,7011309171$ & $2,429132827$ & $90$ \\ \hline
\end{tabular}
\end{table}

%\subsubsection*{Test ${\chi}^{(2)}$ sulle velocità}
% 1b test chi quadro  PDF
%ovviamente ono quele PARI

%test chi quadro su velocità pari con errore corretto  oppure su tutte con err originario da propagazione tempi
%Si è pertanto eseguito un fit lineare sulle $v_{i}$ ad indice PARI e successivamente si è verificata l'ipotesi nulla di un andamento lineare tramite il test del chi %quadro con un livello di confidenza del 99.5\%. Da questo test è emerso che per le sfere 7, 8, e 9 l'ipotesi veniva rigettata. Si è pertanto dedotto che in questi %casi non fosse trascurabile il carattere esponenziale della legge del moto. Vengono riportati i risultati del test in Tabella \ref{tab:TABELLA CI QUADRO VELOCITÀ}.\\
\newpage%
\subsubsection*{Stima del tempo caratteristico $\tau$}
%2 PDF
\begin{wraptable}{l}{4cm}
\centering
    \begin{tabular}{|c|c|}
        \hline
        \textbf{N pallina} & \textbf{$\tau [\si{\milli\second}]$} \\ \hline
        \rowcolor[rgb]{0.85,0.85,0.85}$1$ & $0.148$ \\ \hline
        $2$ & $0.166$ \\ \hline
        \rowcolor[rgb]{0.85,0.85,0.85}$3$ & $0.264$ \\ \hline
        $4$ & $0.396$ \\ \hline
        \rowcolor[rgb]{0.85,0.85,0.85}$5$ & $0.721$ \\ \hline
        $6$ & $1.164$ \\ \hline
        \rowcolor[rgb]{0.85,0.85,0.85}$7$ & $1.830$ \\ \hline
        $8$ & $3.169$ \\ \hline
        \rowcolor[rgb]{0.85,0.85,0.85}$9$ & $6.580$ \\ \hline
        $10$ & $6.320$ \\ \hline
        \rowcolor[rgb]{0.85,0.85,0.85}$11$ & $12.720$ \\ \hline
    \end{tabular}
    \caption{Tempo $\tau$}
    \label{tab:tau}
    \vspace{0.5cm}
\end{wraptable}

Per verificare se la velocità limite fosse stata raggiunta si è stimato il tempo caratteristico $\tau$. L'equazione del moto infatti suggerisce che in un tempo pari a $3\tau$ venga raggiunto il $\approx 95\%$ della velocità limite.
Si è deciso di stimare un tempo caratteristico indicativo, infatti si è assunto che $\eta$ fosse costante lungo la colonna di fluido, ma soprattutto in prima approssimazione si è assunto che la velocità limite fosse la media delle velocità statisticamente indipendenti.
Si è calcolato $\tau$ tramite la seguente:
\begin{equation*}
    \tau = \frac{v_{lim} \cdot \rho_{S}}{g \cdot (\rho_{S} - \rho_{L})}
\end{equation*}
In Tabella \ref{tab:tau} non viene riportato l'errore del tempo caratteristico in quanto non necessario ai fini dell'esperimento e della valutazione di questa verifica. Si è riscontrato che le distanze percorse in $3\tau$, ottenute in modo approssimato da $\Delta x=v_{lim} \cdot 3 \tau$, risultano sempre minori della distanza fra la superficie del liquido e la prima tacca incisa. Il più grande $\Delta x$ ottenuto è $\approx \SI{4}{\milli\metre}$, ciò permette di ipotizzare che in tutti i lanci la velocità limite è stata sempre raggiunta entro lo spazio tra la superficie del fluido e la prima tacca.% e che le velocità nei Grafici \ref{fig:vel1} \dots \ref{fig:vel9} rappresentano le velocità limite.


\subsubsection*{Variazione percentuale delle velocità}
%3 PDF
Per verificare l'ipotesi affermata precedentemente, ovvero il raggiungimento della velocità limite per tutti i viscosimetri, si è scelto di calcolare per ciascuno di questi quanto percentualmente mancasse a ciascuna delle velocità statisticamente indipendenti al raggiungimento di $v_{lim}$, variazione  che secondo l'equazione del moto è pari a:
\begin{equation*}
    \frac{v_{lim} - v_{i}}{v_{i}}=\frac{v_{lim}}{v_i}-1=\frac{a_i}{g' - a_i}
\end{equation*}
Nella formula $g'=g\cdot\frac{\rho_s-\rho_l}{\rho_l}$ è una costante ricavata dall'equazione del moto e le singole accelerazioni $a_i$ sono state calcolate prendendo in considerazione le velocità $v_{i}$ indipendenti e i relativi tempi intermedi $t_{i}$ tramite la seguente:
\begin{equation*}
    a_i=\frac{v_{i+1}-v_i}{t_{i+1}-t_{i}}
\end{equation*}
Si sono calcolati gli errori su $a_{i}$ tramite la formula di propagazione degli errori casuali, in quanto necessari in una successiva fase dell'analisi.\newline
Il valore percentuale che descrive quanto manca alle singole velocità per raggiungere la velocità limite varia tra un valore minimo di $\approx\num{1e-7}\%$ fino ad un valore massimo di $\approx\num{1e-3}\%$. 

Il confronto tra quest'ultimo metodo ed il computo di $\tau$ permette di affermare che le velocità limite sono state raggiunte. Si rimanda alla discussione per una trattazione più approfondita.

\subsection{Stima di $v_{lim}$ per ogni viscosimetro}
Avendo affermato che la $v_{lim}$ è stata raggiunta per tutti i viscosimetri, si è proceduto al computo della miglior stima possibile di quest'ultima, valutando accuratamente l'ipotesi suggerita dal test di t di Student effettuato all'inizio dell'analisi dati.


\subsubsection*{Compatibilità tra singole accelerazioni $a_i$ e accelerazione $a_{tot}$ durante l'intero lancio}

%\begin{wraptable}{r}{8cm}
%\footnotesize
%    \centering
%    \begin{tabular}{|c|c|c|c|}
%        \hline
%        \textbf{Visc.} & \textbf{Acc. Generale} & \textbf{Acc. specifica} & \textbf{$\lambda$} \\ \hline
%        \multirow{4}{*}{1}& \multirow{4}{*}{$(-11 \pm 3)\cdot10^{-11}$}& $(-1 \pm 2)\cdot10^{-10}$ & $2.9$ \\
%        & & $(-3 \pm 2)\cdot10^{-10}$ & $4.1$ \\
%        & & $(-0.2 \pm 2)\cdot10^{-10}$ & $2.0$ \\
%        & & $(-0.2 \pm 2)\cdot10^{-10}$ & $2.0$ \\
%        \hline
%        \multirow{4}{*}{2}& \multirow{4}{*}{$(-16 \pm 8)\cdot10^{-11}$}& $(2 \pm 3)\cdot10^{-10}$ & $0.2$ \\
%        & & $(-7 \pm 3)\cdot10^{-10}$ & $5.5$ \\
%        & & $(0.6 \pm 3)\cdot10^{-10}$ & $1.1$ \\
%        & & $(-0.6 \pm 3)\cdot10^{-10}$ & $1.8$ \\
%        \hline
%        \multirow{4}{*}{3}& \multirow{4}{*}{$(-17 \pm 5)\cdot10^{-11}$}& $(2 \pm 7)\cdot10^{-10}$ & $3.0$ \\
%        & & $(-4 \pm 7)\cdot10^{-10}$ & $6.1$ \\
%        & & $(-3 \pm 7)\cdot10^{-10}$ & $5.8$ \\
%        & & $(-0.3 \pm 7)\cdot10^{-10}$ & $4.2$ \\
%        \hline
%        \multirow{4}{*}{4}& \multirow{4}{*}{$(-3 \pm 3)\cdot10^{-10}$}& $(-0.8 \pm 2)\cdot10^{-9}$ & $5.9$ \\
%        & & $(2 \pm 2)\cdot10^{-9}$ & $0.1$ \\
%        & & $(-2 \pm 2)\cdot10^{-9}$ & $8.5$ \\
%        & & $(-0.2 \pm 1)\cdot10^{-9}$ & $4.2$ \\
%        \hline
%        \multirow{4}{*}{5}& \multirow{4}{*}{$(-3 \pm 1)\cdot10^{-09}$}& $(-11 \pm 5)\cdot10^{-9}$ & $5.8$ \\
%        & & $(0.8 \pm 5)\cdot10^{-9}$ & $1.5$ \\
%        & & $(2 \pm 5)\cdot10^{-9}$ & $1.3$ \\
%        & & $(-5 \pm 5)\cdot10^{-9}$ & $3.7$ \\
%        \hline
%        \multirow{4}{*}{6}& \multirow{4}{*}{$(-7 \pm 2)\cdot10^{-09}$}& $(0.9 \pm 1)\cdot10^{-8}$ & $0.7$ \\
%        & & $(-2 \pm 2)\cdot10^{-8}$ & $6.1$ \\
%        & & $(-0.3 \pm 1)\cdot10^{-8}$ & $2.5$ \\
%        & & $(-0.2 \pm 1)\cdot10^{-8}$ & $2.3$ \\
%        \hline
%        \multirow{4}{*}{7}& \multirow{4}{*}{$(-3 \pm 2)\cdot10^{-08}$}& $(-8 \pm 4)\cdot10^{-8}$ & $3.0$ \\
%        & & $(8 \pm 6)\cdot10^{-8}$ & $0.6$ \\
%        & & $(-20 \pm 7)\cdot10^{-8}$ & $6.9$ \\
%        & & $(11 \pm 6)\cdot10^{-8}$ & $1.4$ \\
%        \hline
%        \multirow{4}{*}{8}& \multirow{4}{*}{$(-32 \pm 5)\cdot10^{-08}$}& $(-6 \pm 3)\cdot10^{-7}$ & $2.9$ \\
%        & & $(-3 \pm 4)\cdot10^{-7}$ & $2.1$ \\
%        & & $(-0.9 \pm 2)\cdot10^{-7}$ & $1.0$ \\
%        & & $(-4 \pm 2)\cdot10^{-7}$ & $1.6$ \\
%        \hline
%        \multirow{4}{*}{9}& \multirow{4}{*}{$(-1 \pm 1)\cdot10^{-6}$}& $(-9 \pm 1)\cdot10^{-6}$ & $6.4$ \\
%        & & $(-8 \pm 7)\cdot10^{-7}$ & $1.0$ \\
%        & & $(-6 \pm 6)\cdot10^{-7}$ & $0.8$ \\
%        & & $(13 \pm 6)\cdot10^{-7}$ & $0.5$ \\
%        \hline
%    \end{tabular}
%    \caption{Accelerazioni $a_{tot}$, $a_{i}$ e $\lambda$}
%    \label{tab:comp_acc}
%\end{wraptable}

Sono state calcolate le accelerazioni $a_{tot}$ relative a ciascun lancio ottenute dal coefficiente angolare della retta interpolante le velocità statisticamente indipendenti dei grafici \textbf{t} vs $\textbf{v}_{i=2n+1}$, associandovi l'errore del coeffciente angolare posteriori.
%L'errore relativo ai tempi intermedi di ciascun intervallo è stato calcolato come propagazione tra i tempi $t_{i}$ e $t_{i+1}$ ottenuti dalle misurazioni al passaggio della sferetta in prossimità dei traguardi. In particolare per i viscosimetri 6 e 9 si è dapprima calcolato l'errore su ciascun $t_i$ come deviazione standard della media tra i tempi ottenuti da ciascun operatore in corrispondenza dell'i-esima tacca. NON È RILEVANTE PER L'ANALISI

Vengono riportate in Tabella \ref{tab:comp_acc} le compatibilità calcolate tra i coefficienti angolari $a_{tot}$ e le singole  accelerazioni $a_{i}$ relative gli intervalli non consecutivi per ciascun viscosimetro.

Si osserva un andamento generale delle $\lambda$ variabile da viscosimetro a viscosimetro, suggerendo l'ipotesi che ogni sfera sia sottoposta a variazioni di velocità non facilmente prevedibili dalla sola equazione del moto, ma causate da variazioni di alcune proprietà fisiche del mezzo viscoso.

\begin{table}[h!]
    \centering
    \footnotesize
    \makebox[\textwidth]{
    \begin{tabular}{|c|c|c|c||c|c|c|c|}
        \hline
        \textbf{Visc.} & \textbf{Acc. Generale} & \textbf{Acc. specifica} & \textbf{$\lambda$} & \textbf{Visc.} & \textbf{Acc. Generale} & \textbf{Acc. specifica} & \textbf{$\lambda$} \\ \hline
        \multirow{4}{*}{1}& \multirow{4}{*}{$(-11 \pm 3)\cdot10^{-11}$}& {\cellcolor[rgb]{0.85,0.85,0.85}}$(-1 \pm 2)\cdot10^{-10}$ & {\cellcolor[rgb]{0.85,0.85,0.85}}$2.9$ & \multirow{4}{*}{6}& \multirow{4}{*}{$(-7 \pm 2)\cdot10^{-9}$}& {\cellcolor[rgb]{0.85,0.85,0.85}}$(0.9 \pm 1)\cdot10^{-8}$ & {\cellcolor[rgb]{0.85,0.85,0.85}}$0.7$\\
        & & $(-3 \pm 2)\cdot10^{-10}$ & $4.1$ & & & $(-2 \pm 2)\cdot10^{-8}$ & $6.1$ \\
        & & {\cellcolor[rgb]{0.85,0.85,0.85}}$(-0.2 \pm 2)\cdot10^{-10}$ & {\cellcolor[rgb]{0.85,0.85,0.85}}$2.0$ & & & {\cellcolor[rgb]{0.85,0.85,0.85}}$(-0.3 \pm 1)\cdot10^{-8}$ & {\cellcolor[rgb]{0.85,0.85,0.85}}$2.5$ \\
        & & $(-0.2 \pm 2)\cdot10^{-10}$ & $2.0$ & & & $(-0.2 \pm 1)\cdot10^{-8}$ & $2.3$ \\
        \hline
        \multirow{4}{*}{2}& \multirow{4}{*}{$(-16 \pm 8)\cdot10^{-11}$}& {\cellcolor[rgb]{0.85,0.85,0.85}}$(2 \pm 3)\cdot10^{-10}$ & {\cellcolor[rgb]{0.85,0.85,0.85}}$0.2$ & \multirow{4}{*}{7}& \multirow{4}{*}{$(-3 \pm 2)\cdot10^{-8}$}& {\cellcolor[rgb]{0.85,0.85,0.85}}$(-8 \pm 4)\cdot10^{-8}$ & {\cellcolor[rgb]{0.85,0.85,0.85}}$3.0$ \\
        & & $(-7 \pm 3)\cdot10^{-10}$ & $5.5$ & & & $(8 \pm 6)\cdot10^{-8}$ & $0.6$ \\
        & & {\cellcolor[rgb]{0.85,0.85,0.85}}$(0.6 \pm 3)\cdot10^{-10}$ & {\cellcolor[rgb]{0.85,0.85,0.85}}$1.1$ & & & {\cellcolor[rgb]{0.85,0.85,0.85}}$(-20 \pm 7)\cdot10^{-8}$ & {\cellcolor[rgb]{0.85,0.85,0.85}}{\cellcolor[rgb]{0.85,0.85,0.85}}$6.9$ \\
        & & $(-0.6 \pm 3)\cdot10^{-10}$ & $1.8$ & & & $(11 \pm 6)\cdot10^{-8}$ & $1.4$ \\
        \hline
        \multirow{4}{*}{3}& \multirow{4}{*}{$(-17 \pm 5)\cdot10^{-11}$}& {\cellcolor[rgb]{0.85,0.85,0.85}}$(2 \pm 7)\cdot10^{-10}$ & {\cellcolor[rgb]{0.85,0.85,0.85}}$3.0$ & \multirow{4}{*}{8}& \multirow{4}{*}{$(-32 \pm 5)\cdot10^{-8}$}& {\cellcolor[rgb]{0.85,0.85,0.85}}$(-6 \pm 3)\cdot10^{-7}$ & {\cellcolor[rgb]{0.85,0.85,0.85}}$2.9$ \\
        & & $(-4 \pm 7)\cdot10^{-10}$ & $6.1$ & & & $(-3 \pm 4)\cdot10^{-7}$ & $2.1$ \\
        & & {\cellcolor[rgb]{0.85,0.85,0.85}}$(-3 \pm 7)\cdot10^{-10}$ & {\cellcolor[rgb]{0.85,0.85,0.85}}$5.8$ & & & {\cellcolor[rgb]{0.85,0.85,0.85}}$(-0.9 \pm 2)\cdot10^{-7}$ & {\cellcolor[rgb]{0.85,0.85,0.85}}$1.0$ \\
        & & $(-0.3 \pm 7)\cdot10^{-10}$ & $4.2$ & & & $(-4 \pm 2)\cdot10^{-7}$ & $1.6$ \\
        \hline
        \multirow{4}{*}{4}& \multirow{4}{*}{$(-3 \pm 3)\cdot10^{-10}$}& {\cellcolor[rgb]{0.85,0.85,0.85}}$(-0.8 \pm 2)\cdot10^{-9}$ & {\cellcolor[rgb]{0.85,0.85,0.85}}$5.9$ & \multirow{4}{*}{9}& \multirow{4}{*}{$(-1 \pm 1)\cdot10^{-6}$}& {\cellcolor[rgb]{0.85,0.85,0.85}}$(-9 \pm 1)\cdot10^{-6}$ & {\cellcolor[rgb]{0.85,0.85,0.85}}$6.4$ \\
        & & $(2 \pm 2)\cdot10^{-9}$ & $0.1$ & & & $(-8 \pm 7)\cdot10^{-7}$ & $1.0$ \\
        & & {\cellcolor[rgb]{0.85,0.85,0.85}}$(-2 \pm 2)\cdot10^{-9}$ & {\cellcolor[rgb]{0.85,0.85,0.85}}$8.5$ & & & {\cellcolor[rgb]{0.85,0.85,0.85}}$(-6 \pm 6)\cdot10^{-7}$ & {\cellcolor[rgb]{0.85,0.85,0.85}}$0.8$ \\
        & & $(-0.2 \pm 1)\cdot10^{-9}$ & $4.2$ & & & $(13 \pm 6)\cdot10^{-7}$ & $0.5$ \\
        \hline
        \multirow{4}{*}{5}& \multirow{4}{*}{$(-3 \pm 1)\cdot10^{-9}$}& {\cellcolor[rgb]{0.85,0.85,0.85}}$(-11 \pm 5)\cdot10^{-9}$ & {\cellcolor[rgb]{0.85,0.85,0.85}}$5.8$ & \multirow{2}{*}{10}& \multirow{2}{*}{ND}& \multirow{2}{*}{ND}& \multirow{2}{*}{ND}\\
        & & $(0.8 \pm 5)\cdot10^{-9}$ & $1.5$ & & & & \\ \cline{5-8}
        & & {\cellcolor[rgb]{0.85,0.85,0.85}}$(2 \pm 5)\cdot10^{-9}$ & {\cellcolor[rgb]{0.85,0.85,0.85}}$1.3$ & \multirow{2}{*}{11} & \multirow{2}{*}{ND}& \multirow{2}{*}{ND}& \multirow{2}{*}{ND}\\
        & & $(-5 \pm 5)\cdot10^{-9}$ & $3.7$ & & & & \\
        \hline
    \end{tabular}}
    \caption{Accelerazioni $a_{tot}$, $a_{i}$ e $\lambda$}
    \label{tab:comp_acc}
\end{table}


\subsubsection*{Stima di $v_{lim}$ con interpolazione lineare su x vs. t}
Per valutare quantitativamente in che modo la viscosità del fluido variasse lungo tutto il cilindro si è proceduto al computo del test del chi quadro sull'interpolazione lineare dei dati \textbf{x} posizione e \textbf{t} tempo e, successivamente, si è ri-effettuato il test eliminando opportunamente le prime misurazioni.\\


Per ottenere una miglior stima possibile della velocità limite si è dunque valutato il coefficiente angolare della retta interpolante i dati \textbf{x} posizione e \textbf{t} tempo di passaggio riferiti a ciascuna tacca considerando la reiezione effettuata. \newline
Per ogni grafico si sono valutati gli errori percentuali dei tempi e delle posizioni dei traguardi.
Per i primi 6 lanci si è impostato il fit lineare su un grafico \textbf{x} vs. \textbf{t}, nei restanti casi si è eseguito il fit su un grafico \textbf{t} vs. \textbf{x}. La stima della velocità deriva dal calcolo del coefficiente angolare nel primo caso o dal reciproco del coefficiente angolare della retta interpolante i dati nel secondo.\\

In entrambi i casi l'errore sulla velocità è stato ricavato impiegando nelle opportune propagazioni l'errore del coefficiente angolare derivato dall'errore a posteriori.\\


Per il decimo ed undicesimo lancio si è assunta come $v_{lim}$ quella derivante dal rapporto fra spazio percorso nell'intervallo di tempo fra le due misurazioni disponibili. 


%FORSE QUI È BENE METTERE ANCHE LE V_LIM STIMATE?
 



\subsection{Verifica della legge di Stokes - Verifica della dipendenza di $v_{lim}$ e $D^2$}
Prima di poter offrire una stima di $\eta$ si è cercato di verificare la legge di Stokes:
\begin{equation*}
    v_{lim}= \frac{{D}^2g\left(\rho_S - \rho_L\right)}{18 \eta }
\end{equation*}
in cui $D$ è il diametro della sfera, $\rho_S$ la densità del materiale di cui la sfera è composta, $\rho_L$ ed $\eta$ rispettivamente la densità e viscosità del fluido.

\subsubsection*{Calcolo di $\eta$ per tutti i viscosimetri}
Si è realizzato il Grafico \ref{fig:eta} in cui nelle ascisse figurano i diametri delle sfere impiegate per ciascun lancio e nelle ordinate i valori delle viscosità $\eta$ calcolati con l'equazione riportata di seguito:
\begin{equation*}
    \eta= \frac{{D}^2g\left(\rho_S - \rho_L\right)}{18 v_{L}} 
\end{equation*}
L'errore della viscosità è stato calcolato tramite propagazione degli errori casuali.

\vspace{0.5cm}
\begin{figure}[h!]
    \centering
    \makebox[\textwidth]{
    \includegraphics[width=12cm]{ETA_DEFINITIVO_per_latex.pdf}
    }
    \caption{Viscosità $\eta$ in funzione del diametro}
    \label{fig:eta}
\end{figure}

%METTIAMO UN FIT ESPONENZIALE? non funziona ahahahaha =)

\subsubsection*{Verifica della dipendenza di $v_{lim}$ da $D$, calcolo di $\eta$}
Si sono poi realizzati due grafici similari per informazioni contenute. Il primo Grafico \ref{fig:verifica_legge_1} riporta in ascissa i valori del diametro delle sfere al quadrato mentre in ordinata il valore delle $v_{lim}$ stimate. Il Grafico \ref{fig:verifica_legge_2} invece riporta in asse \textbf{x} il reciproco del diametro al quadrato, mentre in ordinata il reciproco della velocità limite. $\eta$ è stato stimato in entrambi i casi a partire dal coefficiente angolare della retta interpolante i dati.\newline
Nel primo caso il coefficiente angolare è stato ricavato dalla seguente
\begin{align*}
\begin{split}
    v_{lim}=D^{2} \cdot \underbrace{\frac{g (\rho_s - \rho_l)}{18 \eta}}_\text{Coeff. ang.}
\end{split}
\end{align*}
pertanto è stato possibile stimare una $\eta$ media proprio a partire da questi valori e dalla relativa propagazione.

Analogamente, per il secondo metodo si sono usati i valori di $D^{-2}$ e di $v_{lim}^{-1}$ e $\eta$ è stato stimato dal coefficiente angolare.

\begin{align*}
\begin{split}
     \frac{1}{v_{lim}}= \frac{1}{D^{2}} \cdot \underbrace{\frac{18 \eta}{g (\rho_s - \rho_l)}}_\text{Coeff. ang.}
\end{split}
\end{align*}


\begin{figure}[h!]
    \centering
    %\captionsetup[subfloat]{labelformat=empty}
    \caption*{}
    \makebox[\textwidth]{
    \subfloat[\small Fit $v_{lim}$ vs $D^{2}$]{
    \includegraphics[width=8cm]{diam_quad_vel.pdf}
    \label{fig:verifica_legge_1}
    }
    \subfloat[\small Fit $v_{lim}^{-1}$ vs $D^{-2}$]{
    \includegraphics[width=8cm]{nuovoverificalegge3.pdf}
    \label{fig:verifica_legge_3}
    }}
\end{figure}

\begin{figure}
    \centering
    
\end{figure}

Vengono riportati in Tabella \ref{tab:fit} i valori ottenuti da entrambi i fit con le viscosità annesse.

\begin{table}[h!] %ATTENZIONE ALLE UNITA MISURA
\centering
    \begin{tabular}{|c|c|c|} \hline
        & Fit $v_{lim}$ vs $D^{2}$ & Fit $v_{lim}^{-1}$ vs. $D^{-2}$ \\ \hline
        \rowcolor[rgb]{0.85,0.85,0.85}Intercetta & $-0.0002 \pm 0.006$ & $-27 \pm 2$ \\ \hline
        Coeff. ang. & $0.0006 \pm 0.0002$ & $1880 \pm 10$ \\ \hline
        \rowcolor[rgb]{0.85,0.85,0.85}$\eta$ $[\si{\kilo\gram\per\metre\per\second}]$ & $6 \pm 2$ & $7.01 \pm 0.04$ \\ \hline
    \end{tabular}
\caption{Stime risultanti dai fit}
\label{tab:fit}
\end{table}


\section{Discussione}
\subsection{Errori misurazioni temporali $t_{i,j}$}
Per la presa dati si è deciso di effettuare le misurazioni temporali in corrispondenza del momento in cui il centro della sfera attraversava la superficie identificata dalla tacca segnata sul cilindro. Questo protocollo è stato seguito da tutti e 3 gli operatori nonostante la diversa tecnica utilizzata, con cronometro o tramite software.
Per tutte le misurazioni effettuate tramite il metodo frame-by-frame si sono valutati accuratamente il numero di frame di indecisione entro il quale il centro della sfera avesse raggiunto il livello della tacca incisa.
Le differenti fonti di incertezza si identificano in
\begin{itemize}
    \item Errore sistematico del software/estensione impiegata $\approx 2 $ frame
    \item Incertezza dovuta allo spessore della tacca
    \item Errore di parallasse dovuto al mancato allineamento fra telecamera e tacca, variabile a seconda dei video e delle tacche considerate
    \item Indecisione sull'esatto frame di passaggio nel momento in cui sia possibile una scelta fra diversi fotogrammi
    \item Mancanza di un esatto frame in cui si osserva il passaggio della sfera per la tacca considerata
\end{itemize}
Le prime 4 sfere presentano un elevato numero di frame di incertezza a causa della presenza di tutte le prime quattro componenti di errore, per la ridotta dimensione della sfera e della sua bassa velocità.
Le sfere a diametro intermedio, ovvero dalla quinta all'ottava, presentano come causa d'errore sui frame  sia il mancato allineamento fra obiettivo della telecamera  e la tacca incisa, sia un'incertezza sull'esatto frame di passaggio del centro della sfera per il livello segnato in quanto la massa risulta per diversi fotogrammi consecutivi oscurata dalla tacca. \\


\begin{wrapfigure}{l}{7cm} %DA METTERE PICCOLO A FIANCO A NON ROMPRERE I COGLIONI
    \centering
    \caption{Esempio di frame consecutivi, \\ lancio 8, tacca 11}
    \subfloat{
        \includegraphics[width=0.2\textwidth]{frame1.jpg}
        \label{fig:frame1}
    }
    \subfloat{
 \includegraphics[width=0.2\textwidth]{frame2.jpg}
        \label{fig:frame2}
    }
\end{wrapfigure}
Le sfere a diametro maggiore invece riportano come causa dei frame di incertezza l'errore sistematico del software, l'errore di parallasse e la mancanza del frame rispettivo al momento esatto del passaggio per l'elevata velocità di discesa nel fluido.
In mancanza di un frame ritenuto sufficientemente buono dall'operatore, che mostrasse il passaggio del centro della sfera per la tacca, si è ricorso alla media fra i tempi dei due frame più verisimili. Il valore $t_i$ è talvolta una media fra due tempi e conseguentemente l'errore ad esso associato è maggiore poiché derivante dal teorema delle varianze. L'utilizzo della media fra tempi di frame è stato necessario soltanto nel nono lancio in corrispondenza della tacca 2, 8, 9, 10 ed 11. Si può osservare il fenomeno appena descritto in Figura \ref{fig:frame1} .

Viene riportato il numero opportunamente scelto di frame di incertezza in Tabella \ref{tab:frame_incertezza}, impiegato poi per ricavare $\sigma_{t_{i}}$ tramite la distribuzione triangolare con PTL = \# frame/ framerate $\cdot 1000$ e con coefficiente di affidabilità pari a $1$.\newline


\begin{wraptable}{r}{5cm}
\centering
\caption{Frame di incertezza}
\label{tab:frame_incertezza}
\begin{tabular}{|c|c|}
\hline
Lancio & \# frame  \\ \hline
{\rowcolor[rgb]{0.85,0.85,0.85}}1      & $9$ \\ \hline
2      & $8$ \\ \hline
{\rowcolor[rgb]{0.85,0.85,0.85}}4      & $4$ \\ \hline
5      & $3$ \\ \hline
{\rowcolor[rgb]{0.85,0.85,0.85}}6      & $3$ \\ \hline
9      & $2$ \\ \hline
{\rowcolor[rgb]{0.85,0.85,0.85}}10     & $2$ \\ \hline
11     & $2$ \\ \hline
\end{tabular}
\end{wraptable}

Per le misurazioni effettuate tramite cronometro invece si è ritenuto che le possibili cause di errore fossero:
\begin{itemize}
    \item Imprecisioni dovute alla velocità di risposta dell'operatore nel avviare/bloccare il cronometro
    \item Indecisioni dovute al mancato allineamento della telecamera con la tacca incisa
\end{itemize}
Ogni misurazione risultava avere come incertezza la deviazione standard sulla media relativa tutto il campione di misure riferito ad un unica tacca. Come errore associato alla prima tacca si è considerata la media fra i valori degli errori relativi a tutte le tacche.
Seppur non sia un procedimento statisticamente corretto è ragionevole assumere che l'incertezza sulla prima misurazione sia comparabile all'errore delle misurazioni successive, prestando particolare attenzione nel caso in cui ci fossero dei valori di incertezza per alcune tacche di molto superiori alla media e dunque da escludere per questo computo.


\subsection{Calcolo errore in $\Delta t$}
%Citare il discorso di desalvador, perche un metodo o l'altro, evitare covarianza, citare standard JCGM, uso di media varianze
Per l'analisi dei dati si è scelto di analizzare i $\Delta t$. Così facendo si è evitato di dover rapportare i vari tempi ad un unico zero iniziale in quanto i due metodi impiegati usano origini temporali diverse: l'origine temporale per il metodo con il cronometro corrisponde al passaggio della sfera per la prima tacca mentre per il metodo frame by frame coincide con l'inizio del video.\\
La valutazione dell'errore di $\Delta t$ per le misure effettuate tramite software hanno sfruttato la propagazione sui frame di incertezza, mentre per le misure con cronometro, come consigliato dalla guida JCGM, si sono dapprima calcolati i cinque $\Delta t$ relativi ad uno stesso intervallo fra tacche per poi considerare come componente di errore la loro deviazione standard sulla media.

%In secondo luogo la motivazione è invece da attribuire alla valutazione dell'errore sullo zero dei tempi.In entrambi i metodi sappiamo che lo zero dei tempi è affetto da errore, ma, per il metodo frame by frame, risultava difficile attribuire un errore a quest'ultimo, in quanto non consapevoli dell'esatto funzionamento dell'estensione utilizzata. Si è evitato il problema, in quanto non necessario ai fini dell'esperimento, tramite l'analisi dei $\Delta t$.


\subsection{Commenti sul confronto fra  le metodologie degli operatori - Viscosimetro 6 e 9}
I dati riportati nelle Tabelle \ref{tab:Confronto_tutte_le_compatibilità_per_ogni_misura} e  \ref{tab:Confronto_tutte_le_compatibilità} confermano che ogni campione e ogni singola misura di $\Delta t$ è compatibile con le corrispondenti degli altri operatori. In particolare i campioni tra loro hanno, mediamente, tutti una compatibilità ottima, il che legittima i passaggi descritti nell'analisi.\\
Osservando la tabella \ref{tab:Confronto_tutte_le_compatibilità_per_ogni_misura} si nota inoltre che le compatibilità tra operatore C e operatore A e B per il primo intervallo risultano rispettivamente discrete e pessime. Dal grafico \ref{fig:andamento_delta_t} è osservabile che $\Delta t^C_1$ risulta essere inferiore ai corrispondenti. Si ipotizza quindi un ritardo nell'avvio del cronometro al passaggio della sferetta presso il primo traguardo.\\
%Le misure relative agli intervalli 4, 6 e 10 hanno compatibilità più grande ed errore maggiore  
Generalmente è possibile affermare che le misure dell'operatore A e B risultano più simili tra loro e con compatibilità ottime, oltre che affette da errore minore rispetto a quelle effettuate dall'operatore C. Ciò è giustificabile dall'utilizzo del medesimo metodo di misura e dal fatto che generalmente il metodo frame-by-frame risulti più preciso del metodo con misure tramite cronometro.\newline
Le barre d'errore al quarto e quinto intervallo risultano maggiori delle restanti. Si ipotizza che ciò sia dovuto alla mancanza di precisione dell'operatore C causata da una maggior indecisione nel fermare il cronometro. Invece la compatibilità più alta per le tacche 4, 6 e 10 fra le misure del terzo operatore e gli operatori A e B è ricondotta alla mancanza di accuratezza dell'operatore C, forse implicata al mancato allineamento di obiettivo della telecamera e tacca incisa o a qualche errore di parallasse.\\
L'errore sistematico del software è invece riscontrabile negli scarti fra misurazioni del primo e del secondo operatore, che non risultano mai inferiori a $\approx \SI{33}{\milli\second}$, pari alla durata di un frame.\\
%Probabilmente con un numero maggiore di misure ripetute, il metodo cronometro sarebbe stato più preciso

Al fine di generare un unico campione di $\Delta t$ si è scelto di utilizzare la media ponderata per valutare al meglio la stima prestando attenzione alla componete di errore ottenuta dal metodo utilizzato dall'operatore C, ed in quanto le misure fra i tre diversi operatori risultavano indipendenti.\newline

La stessa valutazione sulle compatibilità è ottenibile dal confronto delle misure tra operatori A e B per il nono viscosimetro, giustificando la creazione un unico campione di $\Delta t$.

\subsection{Calcolo delle $v_i$ e verifica del raggiungimento di $v_{lim}$}
%errore stimato su delta x da quale sfera e come
%tmepi intermedi su grafico delle velocità 
%se ci sono dati che oscillano
%fit che si fa su gli indipendenti
Avendo già a disposizione l'errore sui $\Delta t_{i}$, per il calcolo dell'errore sulle velocità i-esime tramite propagazione degli errori si necessitava dell'errore sull'intervallo di spazio percorso. Per stimare $\sigma_x$ si è utilizzato l' errore derivante dalla distribuzione triangolare utilizzando come PTL lo spessore della tacca incisa, grandezza stimata mediante un confronto tra la più grande sfera che veniva totalmente eclissata dalla tacca stessa. Si è pertanto assunto che lo spessore della tacca incisa fosse pari al diametro della sfera numero 2, avente diametro pari a 2''/32. Per stimare infine l'errore $\sigma_{\Delta x}$ si è utilizzata la formula di propagazione degli errori casuali.  Si stima che l'errore su $\Delta x$ è $\approx \si{0.4}{\milli\metre}$.\\

%commentino grafici
\subsubsection*{Commento ai grafici}
Osservando l'andamento generale dei Grafici \ref{fig:vel1}, ...., \ref{fig:vel9}, si nota che le prime 3 misurazioni variano leggermente rispetto all'andamento generale. Nello specifico per il secondo grafico, la seconda misurazione risulta molto minore rispetto al trend generale. Si ipotizza che una variazione così importante sia dovuta al fatto che il software utilizzato per la presa dati in alcune valutazioni puramente casuali sottostimasse il tempo relativo al frame visualizzato.
Si è cercato di ridurre tale componente di errore analizzando le velocità $v_{1, 3, 5, 7, 9}$, ovvero le velocità ad indice dispari in quanto non influenzate drasticamente da tale errore, come si può osservare intuitivamente dai Grafici. Fenomeno analogo si ripresenta per il Grafico \ref{fig:vel5}

\begin{wraptable}{l}{5cm}
    \centering
    \begin{tabular}{|c|c|}
        \hline
        & Dispersione \% \\ \hline
        \rowcolor[rgb]{0.85,0.85,0.85}1& 3.5311 \\ \hline
        2& 5.1784 \\ \hline
        \rowcolor[rgb]{0.85,0.85,0.85}3& 2.1021 \\ \hline
        4& 3.0381 \\ \hline
        \rowcolor[rgb]{0.85,0.85,0.85}5& 5.5499 \\ \hline
        6& 5.1490 \\ \hline
        \rowcolor[rgb]{0.85,0.85,0.85}7& 10.013 \\ \hline
        8& 20.813 \\ \hline
        \rowcolor[rgb]{0.85,0.85,0.85}9& 37.358 \\ \hline
    \end{tabular}
    \caption{Dispersioni\\ Percentuali}
    \label{tab:dispers_perc}
\end{wraptable}


La Tabella \ref{tab:dispers_perc} quantifica la variazione percentuale delle velocità nel corso dell'intero lancio. Questa stima è stata effettuata tramite la seguente equazione:
\begin{equation*}
    \text{Dispersione \%.} = \frac{v_{Max}- v_{Min}}{v_{Min}} \cdot 100
\end{equation*}

Confrontando i Grafici \ref{fig:vel7} e \ref{fig:vel8}, realizzati con i dai dati raccolti dall'operatore C, e le variazione percentuali, si osservano delle significative fluttuazioni delle $v_{i}$.
La variazione percentuale per il nono lancio risulta molto maggiore dei due lanci precedenti, probabilmente a causa della di un errore dovuto alla presa dati del primo tempo di passaggio, e anche alla variazione delle proprietà del fluido, quali sedimentazione del liquido saponoso o presenza di perturbazione nel fluido. Rivalutando tale variazione escludendo la prima velocità si ottiene infatti un valore di $\approx 15.5\%$, comunque comparabile ai lanci 7 ed 8.

Non è possibile compiere inferenze sui dati dei lanci 10 e 11 in quanto non si hanno a disposizione sufficienti dati per valutare l'andamento delle velocità. Si specifica che lo spazio utilizzato per il computo della velocità in tali lanci è di $\SI{450}{\milli\meter}$, ovvero quello che intercorre tra la seconda e l'ultima tacca.\\

%tstudent
\subsubsection*{Commento a t-Student} %CORREGGERE T STUDENT SU INDIPENDENTI
Si è effettuato il test di t-Student sulla stima del coefficiente di Pearson ottenuto dalle misure $v_{i}$ indipendenti considerando come ipotesi nulla $H_{0}$ la loro non-correlazione in funzione dei tempi $t_{i}$ ovvero $\rho=0$.
Assunta valida la legge oraria del moto, qualora l'ipotesi nulla fosse accettata ($|t| < \tau_{2}$), le $v_{i}$ sarebbero risultate statisticamente non-correlate ai $t_{i}$ e dunque nel caso specifico del moto in analisi ciò avrebbe implicato il raggiungimento del valore asintotico della $v_{lim}$ entro gli errori casuali. Viceversa, nel caso di un rigetto dell'ipotesi nulla, si sarebbe dimostrata una correlazione fra le grandezze, dunque una loro dipendenza dal tempo, dunque o un non raggiungimento della $v_{lim}$ o una non uniformità del liquido saponoso.\\
Dopo il calcolo del parametro t si sono ricercati i livelli di confidenza ad essi corrispondenti, ovvero le soglie oltre le quali l'ipotesi sarebbe risultata falsificata, come viene esposto nell'ultima colonna della Tabella \ref{tab:t_student}.
Si osserva che $H_{0}$ è falsificata con livelli di confidenza piuttosto alti e mai inferiori al 60 \% il che suggerisce con buone probabilità che le misure siano correlate e che sia valida una o entrambe delle due ipotesi precedentemente effettuate relative alla velocità limite e non uniformità del mezzo.\newline
I bassi livelli di confidenza per il rigetto di $H_{0}$ in corrispondenza del quarto e settimo lancio fanno sospettare che le velocità siano in questi due casi statisticamente indipendenti dai tempi. La causa del basso livello di confidenza riscontrato nel viscosimetro 7 è legato all'elevata fluttuazione statistica dei dati. Non si possono fare inferenze sul viscosimetro 4 dato l'andamento dei dati.
%Si è scelto arbitrariamente di eseguire il computo del parametro t di Student soltanto sulle velocità indipendenti in quanto si è osservata.

\subsubsection*{Commento a $\tau$}
Per la valutazione di un $\tau$ indicativo si sono fatte alcune considerazioni e approssimazioni necessarie al suo computo. In primo luogo si considera come $v_{lim}$ la media delle velocità indipendenti e in secondo luogo che la densità e le caratteristiche del fluido, quali per esempio la sua viscosità, siano costanti e non presentino fluttuazioni.
Queste considerazioni, pure essendo forti approssimazioni, permettono di affermare che la velocità limite in tutti i lanci è ampiamente raggiunta entro i primi $\SI{100}{\milli\metre}$ che intercorrono fra la superficie del fluido e la prima tacca incisa.\\
Come già anticipato, non si è stimato l'errore di $\tau$ poiché la valutazione del tempo caratteristico fornisce solo una possibile indicazione per verificare se la velocità limite è stata raggiunta.\newline


%variazione percentuale
\subsection{Calcolo di $v_{lim}$}
\subsubsection*{Commento percentuali}
La stima della percentuale di raggiungimento della $v_{lim}$ è stata ricavata dalla legge del moto. I valori ottenuti per ogni singola $v_{i}$ risultano compresi, come ordini di grandezza, tra $\approx \si{+ 1 \cdot 10^{-2}}\%$ e $ \si{- 1 \cdot 10^{-1}}\%$. Considerando gli errori casuali delle quali sono affette le $v_{i}$ si deduce che le $v_{lim}$ siano state raggiunte in ogni lancio e in tutte le misurazioni effettuate. Quanto affermato è in accordo con la precedente stima di $\tau$ e non contraddice la verifica eseguita tramite il test di t-Student.\newline

Si osservi che per il computo degli errori $a_{i}$ si sono impiegati gli errori delle $v_{i}$ e gli errori dei tempi intermedi $t_{Int, i}$ corrispondenti. Per la stima di $\sigma_{t_{Int, i}}$ si è effettuata la propagazione degli tra la  $\sigma{t_{i}}$ e la $\sigma_{t_{i+1}}$ senza considerare la differenza tra queste ultime con lo zero. Ciò è giustificato in quanto, per il computo dell'accelerazione, questa valutazione dell'errore sarebbe risultata superflua poiché sono state analizzate delle differenze tra $t_{Int, i}$ e $t_{Int, i+1}$. \newline


%Si è scelto di non stimare l'errore stimandolo dopo aver ricalcolato i tempi facendo corrispondere il tempo di passaggio per la prima tacca con $\SI{0}{\milli\second}$. Sebbene non sempre statisticamente corretto, specie per le misurazioni prese tramite il metodo frame-by-frame, questa scelta è stata motivata in quanto non avrebbe significativamente cambiato la stima di tale errore. Per il valore di $ t_{Int, i}$ invece la scelta di non rapportare le misure ad una singola origine risulta del tutto ininfluente. \newline

Osservando il Grafico \ref{fig:vel9} è possibile stimare che la percentuale al raggiungimento di $v_{lim}$ per il primo dato è di $\approx -0.1\%$. Questo valore percentuale è il più grande rilevato fra tutti i vari punti. Come già precedentemente affermato si assume che tale valore sia causato da errori casuali legati alla presa dati o ad errori del software in quanto è stato analizzato dall'operatore A e B, e non da una mancanza del raggiungimento di $v_{lim}$.\newline

I valori percentuali inoltre a volte risultano variare anche nel segno. Ciò implica che le velocità a cui corrispondono stime percentuali negative al raggiungimento di $v_{lim}$ devono diminuire per raggiungere il valore asintotico, mentre quelle a stime percentuali positive devono aumentare in modulo. 

Da questa stima inoltre si osserva che i valori percentuali non seguono un andamento decrescente come ci si potrebbe aspettare assumendo corrette l'equazione del moto e un'uniforme viscosità. Questo comportamento pertanto contraddice l'ultima ipotesi.\newline

Una spiegazione possibile di tali fenomeni viene fornita nel prossimo paragrafo. \newline

\subsubsection*{Commento compatibilità accelerazioni}
Come si osserva dai valori di $\lambda$ riportati nella Tabella \ref{tab:comp_acc}, solo una parte delle accelerazioni specifiche risulta compatibile con le accelerazioni generali di ciascun viscosimetro.
Le compatibilità inoltre sembrano non seguire una particolare distribuzione. Difatti questa casualità deriva dalla fluttuazione statistica delle misure dei tempi effettuate dagli operatori, oppure, più probabilmente, da una non omogenea viscosità del liquido.\newline

Ulteriore prova a sostegno della non omogeneità del fluido è che le accelerazioni risultano essere in modulo diverse l'una dall'altra.
Bisogna osservare infine che le accelerazioni generali fra il primo e l'ultimo lancio differiscono di 4 ordini di grandezza. Tali variazioni rimangono comunque trascurabili e si prova a fornire un giustificazione nel paragrafo dedicato alla discussione di $\eta$.\newline

% Ciò è da attribuire al crescente diametro delle sferette utilizzate, dal quale consegue un maggiore volume, una maggiore massa e dunque una maggiore forza peso.

 \begin{wrapfigure}{r}{3.5cm}
    \centering
    \vspace{0.5cm}
    \includegraphics[width=0.2\textwidth]{fram3.jpg}
    \caption{Scia d'aria}
    \label{fig:scia_aria}
\end{wrapfigure}
Da un'analisi più approfondita del fluido contenuto nella colonna di liquido, compiuta prima dell' esecuzione dell'ottavo lancio, si verifica la presenza di alcune bollicine d'aria generate dalla caduta delle sfere precedenti. Si ipotizza che l'interazione delle sfere alteri le caratteristiche del fluido e che dunque l'equazione del moto utilizzata costituisca solo un'approssimazione.



Ulteriore assunzione è che le sfere siano completamente immerse nel fluido, ma ciò non è applicabile per gli ultimi lanci a diametro maggiore. A partire dall'ottavo lancio si forma infatti una scia d'aria a forma di cono dietro la sfera, dove il fluido non viene in contatto con la superficie metallica, come raffigurato in Figura \ref{fig:scia_aria}. 



%interpolazioni x vs t per velocità
\subsubsection*{Interpolazione su distanze e tempi per stima v lim}
Una volta determinata la non omogeneità del mezzo, si è valutata l'ipotesi che nella prima parte del cilindro contenete il liquido saponoso si avesse una viscosità minore rispetto a quella nella parte finale. Tale ipotesi è suggerita dal fatto che le particelle del fluido, essendo soggette alla forza di gravità, hanno la tendenza ad accumularsi nella parte inferiore del cilindro.\newline
Si è cercato di verificare tale ipotesi effettuando varie volte il test del chi quadro sul fit lineare, eseguito sulle grandezze distanza e tempi di passaggio.
Prima di effettuare la stima della bontà della dipendenza lineare di tali grandezze è risultato necessario analizzare gli errori percentuali fra gli spazi percorsi ed i tempi presi in esame. È emerso che nei primi 6 lanci gli errori percentuali sui tempi sono inferiori di quelli delle posizioni, mentre per gli ultimi 3 lanci, ovvero il settimo, l'ottavo ed il nono, gli errori percentuali sulle posizioni risultavano inferiori. Ciò è una diretta conseguenza degli ordini di grandezza dei tempi misurati. Gli errori percentuali sui tempi, pur avendo incertezze paragonabili fra tutti lanci, sono influenzati dalla dall'ordine di grandezza della misurazione temporale alla quale si riferiscono.


%\begin{table}{r}{4cm}
%\caption{Risultato test ${\chi}^{(2)}$ fit \textbf{t} vs. \textbf{x}}
%\label{tab:test_chi_fit_x_t}
%\centering
%    \begin{tabular}{|c|c|c|}
%     \cline{3-3}
%    \multicolumn{2}{c|}{} & ${\chi}^{(2)}$\\ \hline
%    \multirow{6}{*}{\rotatebox[origin=c]{90}{\textbf{x} vs. \textbf{t}}}& %{\cellcolor[rgb]{0.85,0.85,0.85}}$1$ & {\cellcolor[rgb]{0.85,0.85,0.85}}$1.05992$   \\ %\cline{2-3}
    %& $2$ & $4.40331$       \\ \cline{2-3}
    %& {\cellcolor[rgb]{0.85,0.85,0.85}}$3$ & {\cellcolor[rgb]{0.85,0.85,0.85}}$9.75702$       \\ %\cline{2-3}
    %& $4$ & $13.7078$       \\ \cline{2-3}
    %& {\cellcolor[rgb]{0.85,0.85,0.85}}$5$ & {\cellcolor[rgb]{0.85,0.85,0.85}}$18.208$  \\ %\cline{2-3}
    %& $6$ & $6.51956$       \\ \hline
    %\multirow{3}{*}{\rotatebox[origin=c]{90}{\textbf{t} vs. \textbf{x}}}& %{\cellcolor[rgb]{0.85,0.85,0.85}}$7$ & {\cellcolor[rgb]{0.85,0.85,0.85}}$61.4747$\\ \cline{2-3}
    %& $8$ & $4.65756$ \\ \cline{2-3}
    %& {\cellcolor[rgb]{0.85,0.85,0.85}}$9$ & {\cellcolor[rgb]{0.85,0.85,0.85}}$18.2735$       \\ %\hline
    %\end{tabular}
%\end{table}


%tabella chi quadri senza reiezione con x t o t  x giusti
\begin{table}[h!]
\centering
\begin{tabular}{|c|c|c|c|c|}
\cline{3-5}
\multicolumn{2}{c|}{}&$\chi^{(2)}$ non reieiz &GDL & $\chi^{(2)}$ reiet\\ \hline
\multirow{6}{*}{\rotatebox[origin=c]{90}{\textbf{x} vs. \textbf{t}}}&{\cellcolor[rgb]{0.85,0.85,0.85}}1&  {\cellcolor[rgb]{0.85,0.85,0.85}}$45.5415$& {\cellcolor[rgb]{0.85,0.85,0.85}}$8$ & {\cellcolor[rgb]{0.85,0.85,0.85}}$18.0143$\\ \cline{2-5}
&2&  $37.2959$& $6$ & $4.40331$\\ \cline{2-5}
&{\cellcolor[rgb]{0.85,0.85,0.85}}3&  {\cellcolor[rgb]{0.85,0.85,0.85}}$29.3438$& {\cellcolor[rgb]{0.85,0.85,0.85}}$7$ & {\cellcolor[rgb]{0.85,0.85,0.85}}$15.937$\\ \cline{2-5}
&4&  $23.1531$& $8$ & $19.0326$\\ \cline{2-5}
&{\cellcolor[rgb]{0.85,0.85,0.85}}5&  {\cellcolor[rgb]{0.85,0.85,0.85}}$40.8867$& {\cellcolor[rgb]{0.85,0.85,0.85}}$6$ & {\cellcolor[rgb]{0.85,0.85,0.85}}$18.208$\\ \cline{2-5}
&6&  $56.9285$& $6$ & $6.5195$\\ \hline \hline
\multirow{3}{*}{\rotatebox[origin=c]{90}{\textbf{t} vs. \textbf{x}}}&{\cellcolor[rgb]{0.85,0.85,0.85}}7&  {\cellcolor[rgb]{0.85,0.85,0.85}}$63.8622$& {\cellcolor[rgb]{0.85,0.85,0.85}}$9$ & {\cellcolor[rgb]{0.85,0.85,0.85}}$63.8622$\\ \cline{2-5}
&8&  $19.0456$& $8$ & $8.43574$\\ \cline{2-5}
&{\cellcolor[rgb]{0.85,0.85,0.85}}9&  {\cellcolor[rgb]{0.85,0.85,0.85}}$440.377$& {\cellcolor[rgb]{0.85,0.85,0.85}}$6$ & {\cellcolor[rgb]{0.85,0.85,0.85}}$18.2735$\\ \hline
\end{tabular}
\caption{$\chi^2$ confronto senza e con reiezione}
\label{tab:chiquadro}
\end{table}    

Il confronto dei valori generati dal test del $\chi^2$ giustifica la reiezione. Come si osserva in tabella \ref{tab:chiquadro}, dopo l'eventuale reiezione di un numero opportuno dei primi dati, il valore della variabile $\chi^{2}$, risultava generalmente diminuito.

Il criterio per reiettare si è basato sulla ragionevole eliminazione del minor numero di dati al fine di ottenere un valore della variabile chi quadro accettabile, mantenendo il livello di confidenza maggiore possibile. 

Per il settimo lancio, data la forte variabilità statistica dei dati non è stato possibile accettare l'ipotesi di un andamento lineare dei dati. Questo comportamento è attribuibile alla presenza di una grande incertezza dovuta all'errore di parallasse presente in corrispondenza della sesta tacca.

Si è scelto di accettare l'ipotesi con un livello di confidenza pari al $99.5\%$

%il test del chi quadro riduce notevolmente.
È opportuno precisare che la corrispondenza fra $v_{lim}$ e coefficiente angolare è valida solo se i dati seguono un andamento lineare. Se non venisse verificata tale ipotesi non sarebbe corretto impiegare questo approccio, giustificando la reiezione dei dati e la verifica della loro dipendenza lineare.


\subsection{Verifica legge, ipotesi andamenti esponenziali di viscosità e chi su di loro} 

%Commento ai grafici generico, perchè alcuni errori sono piccoli, sfera 10 e 11
%Parametri fit intercetta errore sistematico e coeff ang è viscosità
%CHi quadro su secono e terzo perchè è schifo, sottostima errori e parabola
%verifica legge fallita per chi qaudro troppo elevato
%Ipotesi stima locale per i primi 3, da osservazoine primo grafico
%Esecuzione fit per stima locale accettata, riportare ci qudro e nuova stima di eta
%media e compatibilità per stima locale

\begin{table}[h!]
    \centering
    \begin{tabular}{|c|c|c|}
        \hline
        &$v_{lim}$ vs. $D^{2}$ &$v_{lim}^{-1}$ vs. $D^{-2}$ \\\hline
        \rowcolor[rgb]{0.85,0.85,0.85}Test $\chi^{(2)}$ & $6.965$ & $ 2.79463$\\\hline
        Intercetta & $(-37 \pm 8)10^{-6}$& $ -13 \pm 3$\\\hline
        \rowcolor[rgb]{0.85,0.85,0.85}Coeff. ang. & $(572 \pm 3)10^{-6}$& $ 1829 \pm 8$\\\hline
        $\eta$ & $6.51 \pm 0.03$& $ 6.81 \pm 0.03$\\\hline
        \end{tabular}
    \caption{Stime di $\eta$ da interpolazione, dopo reiezione}
    \label{tab:eta_interpolazione}
\end{table}

Osservando i Grafici \ref{fig:eta}, \ref{fig:verifica_legge_1}, \ref{fig:verifica_legge_3} è possibile valutare se la legge di Stokes risulta in prima analisi verificata, o meno.
Si nota infatti che nel primo grafico le viscosità presentano un andamento non costante in funzione del diametro, al contrario dell'aspettativa teorica. Si ipotizza che questo andamento sia strettamente correlato alla dimensione della sfera, ma anche alla presenza di altre perturbazioni presenti nel fluido, come già descritto nel paragrafo.\newline
Si riscontra che per i primi tre lanci, ovvero quelli a diametro minore di $\SI{2}{mm}$, $\eta$ rimanga sostanzialmente inalterata, mentre a partire dalla quarta misurazione si iniziano a osservare notevoli cambiamenti. 
È importante osservare che per i lanci 10 e 11, la corrispondente stima di viscosità ha un errore relativo molto piccolo, pari a $\approx 0.5 \%$. Ciò è dovuto al metodo con cui si è stimato l'errore su $v_{lim}$: avendo a disposizione soltanto le misurazioni tra tempi di percorrenza fra seconda ed ultima tacca, non è stato possibile applicare lo stesso metodo usato per gli altri lanci e, data la mancanza di dati si suppone che la stima di $\eta$ ottenuta per quei due lanci abbia altre e differenti componenti di errore.
Per le stime di $\eta$ relative al lancio N'6 si ipotizza che la causa di un elevata componente di errore sia dovuta al fatto che è stato l'unico lancio analizzato da tutti e 3 gli operatori e dunque il più soggetto ad errori di imprecisione ed indecisione della presa dati $v_{lim}$.\\

Dei Grafici \ref{fig:verifica_legge_1} e \ref{fig:verifica_legge_2} vengono riportati i parametri dei fit lineari in Tabella \ref{tab:fit}. Si osserva subito i valori generati dal test del $\chi^{(2)}$ sono di almeno 4 ordini di grandezza maggiori di quanto atteso, rigettando l'ipotesi che le misure seguano un andamento lineare. Tale rigetto dell'ipotesi nulla pregiudica la verifica della legge. In particolare nel Grafico \ref{fig:verifica_legge_1}, è visibile un andamento molto discostato dalla retta interpolante.
Osservando i tre grafici nell'insieme infatti si riscontra che soltanto per i primi lanci si hanno valori vicini a quelli attesi.
Dai due fit lineari eseguiti si riscontra inoltre la presenza di un errore sistematico in quanto l'intercetta delle rette non è nulla. Nel primo caso, l'errore percentuale sull'intercetta è estremamente elevato in quanto il metodo molto suscettibile a variazioni nell'intercetta. Analogo discorso vale per l'errore percentuale del coeffiente angolare. Per questa ragione è preferibile l'impiego del secondo fit per valutare la viscosità in quanto esso risente meno delle variazioni generali dei lanci.


Un confronto incrociato dei metodi utilizzati per definire l'andamento dei grafici per la verifica della legge permette di ipotizzare due possibili cause al fine di spiegare i fenomeni descritti. 
La prima ipotesi identifica nella sottostima dell'errore la causa più probabile dei valori ottenuti per i test del chi quadro utilizzati. Una seconda ipotesi invece è che la legge di Stokes non risulta verificata. Al fine di valutare quantitativamente in che modo la seconda ipotesi permettesse di spiegare al meglio l'andamento di valori degli $\eta$ calcolati si è deciso di eseguire un secondo test del chi quadro solamente per i primi 3 dati, i quali sembravano in prima analisi seguire un andamento lineare. Il nuovo valore ottenuto diminuisce di diversi ordini di grandezza rispetto al calcolo precedentemente effettuato, e ciò suggerisce che localmente la legge risulta verificata. A causa del numero esigui di valori considerati per effettuare il test si è pertanto deciso di valutare un metodo differente per definire in che modo si rapportassero tra di loro i primi valori di $\eta$ ottenuti. Si è calcolata infatti la compatibilità tra i primi valori, che vengono riportati nella tabella \ref{}. Si osserva inoltre che effettuando il calcolo di $\lambda$ tra il terzo ed il quarto dato, si è ottenuto un valore di $4,1$. Per valutare al meglio l'ipotesi precedentemente affermati si è proceduto al computo di nuovi fit lineari sui grafici della verifica della legge.




Il Grafico \ref{fig:verifica_legge_1} riporta un repentino aumento della $v_{lim}$ rispetto a quanto stimato dalla retta interpolante generata tramite il metodo del minimo chi quadro. Non accade lo stesso per il Grafico \ref{fig:verifica_legge_2} nel quale invece tutti i punti sembrano essere bene approssimati dalla retta. In entrambi casi il test del chi quadro dà esito negativo in quanto restituisce un valore di ben 4 ordini di grandezza superiore a quanto mediamente atteso. Tale risultato è da imputare nel primo caso ad un andamento non lineare dei dati mentre nel secondo caso ad errori estremamente piccoli, probabilmente frutto di una sottostima.\newline
Nel secondo e terzo grafico la viscosità del fluido viene stimata considerando tutti i contributi dei viscosimetri e viene rappresentato, congiuntamente al primo grafico, il metodo impiegato per verificare la legge di Stokes.
Sulla base di un'attenta osservazione di tutti i Grafici appena citati si è scelto di stimare $\eta$ considerando soltanto i primi tre lanci. Tali sfere infatti sono quelle tra loro compatibili come si riporta in Tabella \ref{tab:comp_eta}

Si confronti la Tabella \ref{tab:fit} per un confronto 





%confronto degli andamenti

%parlare degli errori

%dire che la legge non è verificata, almeno per tutte non è verificata, ma per le prime sì
Si riscontra che per le prime sfere, nello specifico le sfere avente diametro minore o uguale a 2mm, $\eta$ rimanga costante. Tale affermazione risulta giustificata calcolando le compatibilità dei primi 3 valori tra loro che risultano sempre ottime, ed infine la compatibilità fra terzo e quarto che risultano tra loro non compatibili. Questa affermazione trova ulteriore fondamento nel calcolo del test del chi quadro

%dire che è verificata solamente per i primi 3 viscosimetri

%trattare delle cazzate varie dei regimi turbinosi e laminari ecc...

\begin{table}[h!]
    \centering
    \begin{tabular}{|c|c|}
        \hline
        & $\lambda$\\ \hline
        {\rowcolor[rgb]{0.85,0.85,0.85}}$1-2$ & $0.136661$\\ \hline
        $2-3$ & $0.667946$\\ \hline
        {\rowcolor[rgb]{0.85,0.85,0.85}}$1-3$ & $0.794965$\\ \hline
        $3-4$ & $4.16755$\\ \hline
    \end{tabular}
    \caption{Compatibilità Viscosità}
    \label{tab:comp_eta}
\end{table}


\section{Margini di miglioramento}
%Cronometro con minore input lag
%Errore di parallasse (video poco precisi a volte)

\section{Conclusioni}
%riportare viscosità rappresentativa

\section{Appendice}

\subsection{Formulario}
\textbf{Media, deviazione standard, deviazione standard della media}
\begin{align*}
   % \begin{aligned}
        \overline{x}&=\sum\limits_{i=1}^{N} \frac{x_{i}}{N}&
        \sigma&=\sqrt{\frac{\sum\limits_{i=1}^{N} (x_{i}-\overline{x})}{N-1}}&
        \sigma_{\overline{x}}&=\frac{\sigma}{\sqrt{N}}
   % \end{aligned}
\end{align*}\\

\textbf{Media Ponderata}
\begin{equation*}
\label{eq:media_pond}
    x_i=\frac{\sum_{i=1}^{N}\frac{x_i}{\sigma_{x_i}}}{\sum_{i=1}^{N}\frac{1}{\sigma_{x_i}}}
\end{equation*}

\textbf{Errore Media Ponderata}
\begin{equation*}
\label{eq:errore_media_pond}
     \sigma_{x_i}=\sqrt{\frac{1}{\sum_{i=1}^{N}\frac{1}{\sigma_{i}^{2}}}}
\end{equation*}

\textbf{Formule per il metodo del minimo ${\chi}^{(2)}$}
\begin{equation*}
        \begin{cases}
    a=&\frac{1}{\Delta}[(\sum\limits_{i=1}^{N}{x_{i}^{2}})\cdot(\sum\limits_{i=1}^{N}{y_{i}})-(\sum\limits_{i=1}^{N}{x_{i}})\cdot(\sum\limits_{i=1}^{N}{x_{i}y_{i}})] \\ 
    b=&\frac{1}{\Delta }\cdot \left [N\cdot \left ( \sum\limits_{i=1}^{N}x_i y_i \right )-\left ( \sum\limits_{i=1}^{N}x_i \right )\cdot \left ( \sum\limits_{i=1}^{N}y_i \right )  \right ]\\
    \Delta=& N\cdot \sum\limits_{i=1}^{N} x_i^{2} - \left ( \sum\limits_{i=1}^{N}x_i \right )^{2}\\
    \end{cases}
\end{equation*}
\begin{equation*}
    \begin{cases}
    \sigma_{a}=&\sigma_{y}\cdot\sqrt{\frac{\sum_{i=1}^{N}{x_{i}^{2}}}{\Delta}} \\
    \sigma_{b}=&\sigma_y\cdot \sqrt{\frac{N}{\Delta }}\\
    \end{cases}
    \label{equation:err_chi_quadro}
\end{equation*}
\\
\textbf{Formula di propagazione degli errori casuali}\\

Sia z=($x_1$,...;$x_N$) funzione di N variabili casuali $x_1$,...,$x_N$ e sia ${x_i^\ast}$=($x_1^\ast$,...,$x_N^{\ast}$) l'insieme di tutti i valori veri associati a tali variabili, si ha 

\begin{equation*}
    \sigma_z^{2}\approx  \sum_{i=j=1}^{N}\left ( \frac{\partial z}{\partial x_i}\Big|_{x_i^{\ast}} \right )^{2}\cdot\sigma_{x_i}^{2} +\sum_{i=1,j=1,i\neq j}^{N}\left (\frac{\partial z }{\partial x_i}\Big|_{x_i^{\ast}} \right ) \cdot \left ( \frac{\partial z}{\partial x_j} \Big|_{x_j^{\ast}} \right )\cdot cov(x_i,x_j)\label{eq:prop_errori}
\end{equation*}
E' stato utilizzato il simbolo $\approx$ in quanto si è scelto di troncare al primo termine lo sviluppo in serie di Taylor.\\


\textbf{Formula calcolo compatibilità}\\
\begin{equation*}
    \lambda=\frac{\left|a-b\right|}{\sqrt{\sigma^{2}_{a}+\sigma^{2}_{b}}}
\end{equation*}\\
\textbf{Coefficiente di correlazione di Pearson}\\
\begin{equation*}
    \rho=  \frac{\sum_{i=1}^{N}(x_i - \overline{x}
    )(y_i - \overline{y})}{\sqrt{\sum_{i=1}^{N}(x_i -\overline{x})^2}\sqrt{\sum_{i=1}^{N}(y_i - \overline{y})^2}}
\end{equation*}

\textbf{t-Student su stima di $r$ di Pearson}\\
\begin{equation*}
    t=\frac{r \cdot \sqrt{N-2} }{\sqrt{1- r^2}}
\end{equation*}

\end{document}

